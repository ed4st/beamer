\documentclass{beamer}
\usepackage[utf8]{inputenc}
\usetheme[background=dark,titleformat = smallcaps , block = fill,numbering = fraction, progressbar = 
frametitle , titleformat title= smallcaps]{metropolis}           % Use metropolis theme


\definecolor{orangeBar}{HTML}{FF3600}
\setbeamercolor{progress bar}{fg=orangeBar}

\usepackage{multimedia}
\usepackage{animate}
\usepackage {extarrows}
\usepackage {tikz}

\usepackage[spanish]{babel}
\usepackage{graphicx}
\usepackage{amssymb}
%\usepackage{amsfonts}

\usepackage{enumerate}
\usepackage{amsmath}
\usepackage{amsthm}
\usepackage{xcolor}
%\usepackage{amsfonts,amssymb,amsthm}

\usepackage{url}
\usepackage{enumerate}
\usepackage{commath}
\usepackage{multicol}
\usepackage{mathtools}
\usepackage{scrextend}
\usepackage{hyperref}
\usepackage{cleveref}
\usepackage{longtable}
\usepackage{bbm}
\usepackage{siunitx}
\usepackage{listings}
\usepackage{xcolor}
\usepackage{subcaption}
\usepackage{epigraph}
\usetikzlibrary{arrows}

\definecolor{codegreen}{rgb}{0,0.6,0}
\definecolor{codegray}{rgb}{0.5,0.5,0.5}
\definecolor{codepurple}{rgb}{0.58,0,0.82}
\definecolor{backcolour}{rgb}{0.95,0.95,0.92}
\definecolor{darkBlue}{HTML}{00000F}
\definecolor{lightBlue}{HTML}{00B7D4}
\definecolor{lightRed}{HTML}{E42525}
\definecolor{lightGreen}{HTML}{9CE425}

\definecolor{green0}{HTML}{B65900}
\definecolor{green1}{HTML}{D77200}
\definecolor{green2}{HTML}{ED8E30}
\definecolor{green3}{HTML}{FABE86}

\lstdefinestyle{mystyle}{
    backgroundcolor=\color{darkBlue},   
	commentstyle=\color{lightGreen},
	keywordstyle=\color{lightBlue},
	numberstyle=\tiny\color{codegray},
	stringstyle=\color{lightRed},
	basicstyle=\ttfamily\footnotesize,
	breakatwhitespace=false,         
	breaklines=true,                 
	captionpos=b,                    
	keepspaces=true,                 
	numbers=left,                    
	numbersep=5pt,                  
	showspaces=false,                
	showstringspaces=false,
	showtabs=false,                  
	tabsize=1
}

\lstset{style=mystyle}
\lstset{language=Python}
\lstset{frame=lines}
\lstset{caption={Insert code directly in your document}}
\lstset{label={lst:code_direct}}
\lstset{basicstyle=\footnotesize}


\newcommand{\bb}[1]{\mathbb{#1}}

%\newtheorem{theorem}{Teorema}[section]
%\theoremstyle{plain}
\newtheorem{acknowledgement}[theorem]{Acknowledgement}
\newtheorem{algorithm}[theorem]{Algorithm}
\newtheorem{axiom}[theorem]{Axiom}
\newtheorem{case}[theorem]{Case}
\newtheorem{claim}{Claim}
\newtheorem{conclution}[theorem]{Conclusión}
\newtheorem{condition}[theorem]{Condition}
\newtheorem{conjecture}[theorem]{Conjecture}
%\newtheorem{corollary}[theorem]{Corolario}
\newtheorem{criterion}[theorem]{Criterion}
\theoremstyle{definition}
%\newtheorem*{df}{Definición}
%\newtheorem{definition}[theorem]{Definición}
%\newtheorem{example}[theorem]{Ejemplo}
\newtheorem{exercise}[theorem]{Exercise}
%\newtheorem{lemma}[theorem]{Lema}
\newtheorem{notation}[theorem]{Notation}
%\newtheorem{problem}[theorem]{Problem}
\newtheorem{proposition}[theorem]{Proposición}
\newtheorem{remark}[theorem]{Nota}
%\newtheorem{solution}[theorem]{Solución}
\newtheorem{summary}[theorem]{Summary}
\numberwithin{equation}{section}

\definecolor{defColor}{HTML}{3ED597}
\newcommand{\marine}[1]{\textcolor{defColor}{#1}}


\definecolor{thColor}{HTML}{FA7E0A}
\newcommand{\orangee}[1]{\textcolor{thColor}{#1}}

\definecolor{rkColor}{HTML}{F72121}
\newcommand{\redd}[1]{\textcolor{rkColor}{#1}}


%---------------emojis--------------------
\newcommand{\smiley}{\tikz[baseline=-0.75ex,black]{
		\draw circle (2mm);
		\node[fill,circle,inner sep=0.5pt] (left eye) at (135:0.8mm) {};
		\node[fill,circle,inner sep=0.5pt] (right eye) at (45:0.8mm) {};
		\draw (-145:0.9mm) arc (-120:-60:1.5mm);
	}
}

\newcommand{\frownie}{\tikz[baseline=-0.75ex,black]{
		\draw circle (2mm);
		\node[fill,circle,inner sep=0.5pt] (left eye) at (135:0.8mm) {};
		\node[fill,circle,inner sep=0.5pt] (right eye) at (45:0.8mm) {};
		\draw (-145:0.9mm) arc (120:60:1.5mm);
	}
}

\newcommand{\neutranie}{\tikz[baseline=-0.75ex,black]{
		\draw circle (2mm);
		\node[fill,circle,inner sep=0.5pt] (left eye) at (135:0.8mm) {};
		\node[fill,circle,inner sep=0.5pt] (right eye) at (45:0.8mm) {};
		\draw (-135:0.9mm) -- (-45:0.9mm);
	}
}




\newtheorem{df}{\marine{Definición}}
\newtheorem{thh}{\orangee{Teorema}}
\newtheorem{pr}{\orangee{Proposición}}
\newtheorem{lm}{\orangee{Lema}}
\newtheorem{crr}{\orangee{Corolario}}
\newtheorem{rr}{\redd{Observación}}
\usepackage{graphicx} 


%\newtheorem{defn}[]{Definición}
%\newenvironment{definition}{\begin{defn}}{\end{defn}}
%\newtheorem{definition}{Definition}[section]
%\newtheorem*{remark}{Remark}
%%%%%%%%
\newcommand{\tit}[1]{\textit{#1}}
\newcommand{\bsym}{\mathbf}
\newcommand{\Mod}[1]{\ (\mathrm{mod}\ #1)}
%\newcommand{\blue}[1]{\textcolor{blue}{#1}}
\newcommand{\red}[1]{\textcolor{red}{#1}}
\renewcommand{\geq}{\geqslant}
\renewcommand{\leq}{\leqslant}
\newcommand{\Rplus}{\mathds{R}_{^{+}}}
\newcommand{\N}{\mathbb{N}}
\newcommand{\Z}{\mathbb{Z}}
\newcommand{\R}{\mathbb{R}}

\newcommand{\C}{\mathbb{C}}
\newcommand{\Q}{\mathbb{Q}}
\newcommand{\ssi}{\longleftrightarrow}
\newcommand{\ent}{\longrightarrow}
\newcommand{\Qp}{\mathbb{Q}_p}  
\newcommand{\Qpn}{\mathbb{Q}_p^n}
\newcommand{\Zpn}{\mathbb{Z}_p^n}
\newcommand{\Zp}{\mathbb{Z}_p}
\newcommand{\Zd}{\mathbb{Z}_2}
%\newcommand{\abs}[1]{\left\vert #1 \right\vert}
%\newcommand{\norm}[1]{\|#1\|}
\newcommand{\pnorm}[1]{\|#1\|_p}
\newcommand{\maxx}[1]{\text{m\'ax} #1}
\newcommand{\xbar}[1]{\hskip 1.4pt\overline{\hskip-1.2pt #1\hskip -.6pt}\hskip 1.2pt}
\newcommand{\rb}{\raisebox{-.35ex}}

\DeclareMathOperator{\s}{\mathbf{S}}
\DeclareMathOperator{\f}{\mathcal{F}}
\DeclareMathOperator{\A}{\mathbb{A}}
\DeclareMathOperator{\dist}{dist} % The distance.
\DeclareMathOperator{\d^n}{\dif^{\,n}}
%\DeclareMathOperator{\d}{\dif}
\DeclareMathOperator{\Real}{Re}
\DeclareMathOperator{\ord}{Ord}
\DeclareMathOperator{\Dom}{Dom}
\DeclareMathOperator{\vol}{vol}
\DeclareMathOperator{\gpn}{\mathit{{GpnN}}}
%%%%%%%%%
%%%%%%%%%%%%%%%%% dashed integrals %%%%%%%%%%%%%%%%%%%%%
\DeclareSymbolFont{eulargesymbols}{U}{zeuex}{m}{n}
\DeclareMathSymbol{\intop}{\mathop}{eulargesymbols}{"52}
\usepackage[toc,page]{appendix}
\renewcommand{\labelitemi}{$\circ$}


\title{Una introducción a los números $p$-ádicos, su aritmética y algunas simulaciones en \textit{Python}}
\subtitle{Trabajo de grado presentado para optar por el título de
Matemático}
\date{\today}


\author{\bf{Autor: }Edgar Steven Baquero Acevedo
	\\ \bf{Supervisor: }Leonardo Fabio Chacón Cortés, Ph. D.}

%\institute{Pontificia Universidad Javeriana,
%Facultad de Ciencias\\
%Departamento de Matemáticas}



\usepackage{MnSymbol,wasysym}



\titlegraphic{%
	\begin{picture}(0,0)
	\put(200,-200){\makebox(0,0)[rt]{\includegraphics[width=3cm]{img/logo_Ujaveriana}}}
	\end{picture}}

\begin{document}
  \maketitle

\begin{frame}{Motivación}
	
	\begin{figure}
		
		\includegraphics[width=4cm]{img/kurt_hensel.jpg}
		\caption{\textit{Kurt Hensel}}
	\end{figure}

\end{frame}



\begin{frame}{Contenido}
	\tableofcontents
\end{frame}

  \section{Notación}

  \begin{frame}{Unidades, decenas y centenas...}
    %En la escuela nos enseñaron a separar los números por unidades, decenas y centenas. Por ejemplo el número $437$ tiene $7$ unidades, $3$ decenas y $4$ centenas. Es decir que podemos representar $437$ como:
    %$$437 = 7\cdot10^0+ 3\cdot 10^1 + 4\cdot 10^2,$$
    %El número $543.89$ como:
    %$$543.89 = %9\cdot10^{-2}+8\cdot10^{-1}+3\cdot10^0+4\cdot10^1+5\cdot10^2.$$
    %Que también se puede denotar como $543.89_{10}$.
    
    \begin{center}
    	\animategraphics[autoplay, every = 2, width=1.0\linewidth]{30}{Notacion_1/notacion00}{001}{332}
    \end{center}
  \end{frame}

  \begin{frame}{Sistemas numéricos}
  	Así:
  	\begin{itemize}
  		\item Se puede expandir un número por cualquier base $q$.
 		\item Ejemplos conocidos de sistemas de numeración son el \tit{octal, hexadecimal y binario}, entre otros.
  	\end{itemize}
  \begin{exampleblock}{Ejemplo}
	Podemos representar el siguiente número:
	$$2\cdot8^{-2}+2\cdot8^{-1}+3\cdot8^0+4\cdot8^1+7\cdot8^2,$$
	por $743.22$ ($q=8$), o también $743.22_8$.
  \end{exampleblock}
	 
  \end{frame}

%3------------------
\begin{frame}{Representación que usamos}
	\begin{itemize}[<+- | alert@+>]
		\item Particularmente, estamos interesados en expansiones sobre bases primas.
		\item Por ejemplo con $p=2$, el ¡Sistema binario!
		\item En general, una expansión de la forma.	$$x=\sum_{k=-\gamma}^{l}a_kp^k,\text{ con $\gamma\in\Z$, $a_k\in \{0,\dots,p-1\}$ },$$
		será
		\begin{equation}\label{notacion}
		{a_{l} \ldots a_{2} a_{1} a_{0},a_{-1}a_{-2}\cdots a_{-\gamma}}_p.
		\end{equation}
		\item ¿Deberíamos llamar a las expansiones, expansiones pesimales?
		\item Usaremos los términos: expansión (representación) $p$-ádica ó \tit{Código de Hensel}.
		\item Siendo así, ahora sí empecemos.
	\end{itemize}

\end{frame}
%4-------------------
\section{El campo de los números $p$-ádicos}
\begin{frame}{Norma}
\begin{df}\label{pnorm}
	Sea $K$ un cuerpo. Una \textit{norma} en $K$ es una función
	$\abs{\cdot}\colon K \to \R_{\geq 0}$ tal que para todo $x,y\in$ $K$
	satisface las siguientes propiedades:
	
	\begin{enumerate}[<+- | alert@+>]
		\item $\abs{x} \geq0$, $\abs{x}
		=0\Longleftrightarrow x=0,$
		
		\item $\abs{xy}  =\abs{x} \left\vert 
		y\right\vert ,$
		
		\item $\left\vert x+y\right\vert \leq\abs{x} +\left\vert y\right\vert$.
		\item Si $\abs{x+y} \le \maxx\{\abs{x}, \abs{y}\}$, para todo $x,y\in K,$ decimos que la norma es \tit{ultamétrica}, o que satisface la \tit{desigualdad triangular fuerte}.
	\end{enumerate}
\end{df}
	Además, una norma $\abs{\cdot}$ en $K$ define una métrica natural dada por\linebreak ${d (x,y)=\abs{x-y}}$.
\end{frame}
%5-----------------------------------------
\begin{frame}{Equivalencia entre normas}
\begin{df}
	Dos normas $\abs{\cdot}_1$, $\abs{\cdot}_2$ sobre un cuerpo $K$ se dicen \tit{equivalentes }si inducen la misma topología sobre $K$.
\end{df}

\begin{pr}\label{quiv_power}
	Las siguientes condiciones son equivalentes:
	\begin{enumerate}
		\item $\abs{\cdot}_1\sim\abs{\cdot}_2$.
		\item Existen constantes $k_1,k_2$ positivas tales que:
		$$k_1\abs{x}_1\leq\abs{x}_2\leq k_2\abs{x}_1,\text{ para todo $x\in K$.}$$ 
		\item Existe $c\in\bb{R}_{>0}$ tal que $\abs{\cdot}_1=\abs{\cdot}_2^c$.	
		\item Una sucesión $ (x_n)$ es de Cauchy respecto a $\abs{\cdot}_1$ si, y sólo si, es de Cauchy respecto a $\abs{\cdot}_2$.
	\end{enumerate}

\end{pr}
\end{frame}
%6----------------------

%7----------------------


%8----------------------
\begin{frame}{Orden y Norma en $\Q$}
\begin{df} \label{ord_def_1}
	Fijemos un primo $p$, sea $x\in\mathbb{Q\smallsetminus}\left\{  0\right\}  $ expresado de forma única como $x=p^{v}\frac{a}{b}$, donde $v$ es un entero y
	$a$, $b$ son  primos relativos con $p$.
	Definimos la función $\pnorm{\cdot}$ de la siguiente manera:
	\[
	\| x\| _{p}=p^{-v},
	\]
	donde el entero $v=v\left (  x\right)  $ se denomina el orden $p$\textit{-ádico de} $x$ y
	será denotado por $\ord\left (  x\right)  $. Por definición $\|0\|_p=0$, y  $\ord (0)=+\infty $.
\end{df}
\end{frame}
%9----------------------
\begin{frame}{Orden y Norma en $\Q$}
	 \begin{center}
		\animategraphics[autoplay, every = 1, width=1.0\linewidth]{30}{Norma/norma00}{001}{164}
	\end{center}
	%\begin{exampleblock}{Ejemplo}
	%	Cálculo de la función $\pnorm{\cdot}$ para distintos $p$'s.
	%	\[
	%	\left| -\frac{66}{500}\right| _{p}=\left| -\frac{33}{250}\right| _%{p}=\left| -\frac{3\cdot11}{2\cdot5^3}\right| _{p}=\begin{cases}
	%	\frac{33}{250}       &\,\,\, \text{ si } \,\,\,p=\infty;\\ 
	%	2                    &\,\,\, \text{ si }\,\,\, p=2;\\
	%	\frac{1}{3}          &\,\,\, \text{ si } \,\,\,p=3;\\  
	%	5^3                  &\,\,\, \text{ si }\,\,\, p=5;\\   
	%	1                    &\,\,\, \text{ si }\,\,\, p=7;\\    
	%	\frac{1}{11}         & \,\,\,\text{ si }\,\,\, p=11;\\     
	%	& \vdots\\
	%	1                   &\,\,\,\text{ otro caso}.
	%	\end{cases}
	%	\]
	%\end{exampleblock}
\end{frame}
%10-----------------------------------------------
\begin{frame}{Los 3 mosqueteros}
	
	\begin{thh}
		[\tit{Fórmula Adélica del producto}]
		Sea $x\in \bb{Q}$ tal que $x\neq0$, entonces:
		$$\displaystyle\prod_{p}^\infty \norm{x}_p=1, \text{ con } \norm{x}_\infty=\abs{x} \text{ y $p$ primo}.$$
	\end{thh}
\begin{thh}
		$\| \cdot \|_p$ es una norma no arquimediana.	
\end{thh}
\begin{thh}
	[Ostrowski]	\label{ostrowsky} Cualquier norma no trivial sobre $\mathbb{Q}$ es equivalente al
	valor absoluto usual, o a una norma p-ádica $\| \cdot\| _{p}%
	$, para algún primo $p$.
\end{thh}
\end{frame}
%11-----------------------------------------------

%12-----------------------------------------------
\begin{frame}{Sucesiones de Cauchy}

\begin{thh}
	$ (\Q, d (x,y)=\pnorm{x-y})$ y $ (\Q, d (x,y)=|x-y|)$ no son espacios completos \frownie{}.
\end{thh}
\begin{thh}
	$\Qp$ es la completación de $\Q$ con la norma $\pnorm{\cdot}$.
\end{thh}
\begin{thh}
	[Caracterización de sucesiones de Cauchy]\label{car}
	Una sucesión $ (x_n)_{n\in\N}$ en $\Qp$ es de Cauchy, si, y sólo si:
	\begin{equation}\label{car_cau}
	\lim_{n\to\infty}\pnorm{x_{n+1}-x_n}=0.
	\end{equation}
\end{thh}
\end{frame}
%14-----------------------------------------------

\begin{frame}[fragile]{Completaciones de $\Qp$}
	\begin{figure}[h]
		\centering
		
		\begin{tikzpicture}
		% \draw[black!20] (-5,-5) grid (5,5);
		%\draw[orange]
		\node at (0,0) {$\mathbb{Q}$};
		
		\draw[dashed] (0,3) arc (90:225:3);
		
		\draw[-latex] (0,0.707107) -- (0,3);
		\draw   (0,3)   circle (0pt)  node[anchor=south]{$\mathbb{R}$};
		\node at (0.5,1.8) {\tiny{$\norm{\cdot}_{\infty}$}};
		
		\draw[-latex] (0.707107,0) -- (3,0);
		\draw   (3,0)   circle (0pt)  node[anchor=west]{$\mathbb{Q}_3$};
		\node at (1.9,-0.3) {\tiny{$\norm{\cdot}_{3}$}};
		
		
		\draw[-latex] (0,-0.707107) -- (0,-3);
		\draw   (0,-3)   circle (0pt)  node[anchor=north]{$\mathbb{Q}_7$};
		\node at (-0.4,-1.8) {\tiny{$\norm{\cdot}_{7}$}};
		
		\draw[-latex] (0.5,0.5) -- (2.12132034355964,2.12132034355964);
		\draw  (2.42132034355964,2.72132034355964)    circle (0pt)  node[anchor=north]{$\mathbb{Q}_2$};
		\node at (1.5,1) {\tiny{$\norm{\cdot}_{2}$}};
		
		\draw[-latex] (-0.5,-0.5) -- (-2.12132034355964,-2.12132034355964);
		\draw   (-2.42132034355964,-1.92132034355964)   circle (0pt)  node[anchor=north]{$\mathbb{Q}_{11}$};    
		\node at (-1.8,-1.2) {\tiny{$\norm{\cdot}_{11}$}};
		
		\draw[-latex] (0.5,-0.5) -- (2.12132034355964,-2.12132034355964);
		\draw   (2.42132034355964,-1.92132034355964)   circle (0pt)  node[anchor=north]{$\mathbb{Q}_5$};
		\node at (1,-1.5) {\tiny{$\norm{\cdot}_{5}$}};
		
		
		\draw[-latex] (-0.5,0.5) -- (-2.12132034355964,2.12132034355964);
		\draw   (-2.52132034355964,2.92132034355964)   circle (0pt)  node[anchor=north]{$\mathbb{Q}_p$};
		\node at (-1,1.6) {\tiny{$\norm{\cdot}_{p}$}};
		
		
		
		%   \draw[-{Latex}] (-2.82842712475419,2.82842712475419) -- (-3.53553390593275,3.53553390593275);
		%   \draw   (-3.93553390593275,4.33553390593275)   circle (0pt)  node[anchor=north]{$\mathbb{Q}_p(\sqrt{d})$};    
		
		\end{tikzpicture}
		
		\caption{Completaciones respecto a las distintas normas en $\Q$}
		
	\end{figure}
	
	
	%\begin{figure}
		%\includegraphics[scale=0.35]{img/relojBeamer.jpg}\caption{Completaciones respecto a las distintas normas en $\Q$}
 	%\end{figure}
\end{frame}
%15-----------------------------------------------
\iffalse 
\begin{frame}{$\Q$ no es completo \frownie{}}
	El siguiente teorema es importante, pues caracteriza a $\Q$ como un cuerpo no completo.
	\begin{thh}
		$ (\Q, d (x,y)=\pnorm{x-y})$ y $ (\Q, d (x,y)=|x-y|)$ no son espacios completos.
	\end{thh}
\begin{exampleblock}{Ejemplo}
	Un procedimiento para construir una sucesión de Cauchy en $ (\Q, d (x,y)=\pnorm{x-y})$ podría hacerse, tomando $a\in\Q$ tal que:
	\begin{itemize}[<+- | alert@+>]
		\item[$\diamond$] $a$ no es cuadrado en $\Q$
		\item[$\diamond$] $p \nmid a$
		\item[$\diamond$] $a$ es residuo cuadrático módulo $p$. i.e., $x^2 \equiv a \Mod{p^n}$ tiene solución.
	\end{itemize}
\end{exampleblock}
\end{frame}


%16-----------------------------------------------

\begin{frame}{$\Q$ no es completo :c}
	\begin{exampleblock}{Continuación$\dots$}
		Podemos hallar $a$ tal que sea cuadrado en $\Z$ y sumarle un múltplo de $p$; para así construir la sucesión como sigue:
		\begin{itemize}[<+- | alert@+>]
			\item[$\diamond$] Tomamos $x_0$ solución de $x^2\equiv a \Mod{p}$
			\item[$\diamond$] Construimos a $x_1$ tal que $x_1 \equiv x_0 \Mod{p}$ y además ${x_1^2\equiv a \Mod{p^2}}$ 
			\item[$\diamond$] Recursivamente, construimos $x_n$ tal que:
			$$x_n \equiv x_{n-1} \Mod{p^n} \text{ y } {x_n^2\equiv a \Mod{p^{n+1}}}$$
		\end{itemize}
	Es de cauchy: $  \pnorm{x_{n+1}-x_n} = \pnorm{kp^n} \leq \pnorm{p^n}=p^{-n}\rightarrow0.$\linebreak
	No converge: $\pnorm{x_n^2-a}=\pnorm{sp^{n+1}}\leq \pnorm{p^{n+1}}\leq p^{- (n+1)}\to 0,$
	luego $x_n\to \sqrt{a}\notin\Q$.
	\end{exampleblock}
\end{frame}
\fi
%-------------Topology----------------------
\subsection{Topología en $\Qp$}
\section*{Topología en $\Qp$}
%17-----------------------------------------
\begin{frame}{El espacio $\Qpn$}
	Podemos definir en $\Qpn$ una norma como:\[
	\pnorm{x}:=\max_{1\leq i\leq n}\pnorm{x_i},\qquad\text{para }x= (x_{1},\dots,x_{n})\in\Qpn.
	\]
	Así, $(\Qpn, d=\pnorm{x-y})$ es un espacio métrico, donde las distancias están en el conjunto \{$p^\gamma$$\colon$ $\gamma\in\Z$\}$\cup$\{$0$\}. Luego, tiene sentido definir los abiertos básicos por:
	 \[
	 B^n_{\gamma} (a)=\{x\in\Qp:\pnorm{x-a}< p^{\gamma}\},\ \gamma\in \mathbb{Z}.
	 \]
	 \[
	 S^n_{\gamma} (a)=\{x\in\Qpn:\pnorm{x-a}=p^{\gamma}\},\ \gamma\in \mathbb{Z}.
	 \]
	 \begin{rr}
	 	$B_\gamma^n (a)$ es un grupo aditivo.
	 \end{rr}
\end{frame}
%18-----------------------------------------

%18-----------------------------------------
\begin{frame}{Propiedades bonitas de $\Qp$ \smiley{}}
	\begin{thh}
	\begin{itemize}[<+- | alert@+>]
		\item \label{clopen1} Si $b\in B_{r} (a)$, entonces $B_{r} (a)=B_{r} (b)$. En otras palabras: ¡Todo  punto de una bola abierta es centro de la misma!
		\item Toda bola es a su vez, un conjunto cerrado y abierto.
		\item Dos bolas en $\Qp$ son disyuntas o una contiene a la otra; es decir, si $a,b \in \bb{Q}_p$, y $r,s\in \Z$, se tiene que $B_{r} (a)\cap B_{s} (b)\neq\emptyset$ si, y sólo si, $B_{r} (a)\subseteq B_{s} (b)$ o $B_{s} (b)\subseteq B_{r} (a)$.
	\end{itemize}	
	\end{thh}
	
	
\end{frame}

%19-----------------------------------------
\begin{frame}{Más propiedades de $\Qp$ }
\begin{thh}
	$\Qp$ es un espacio de \tit{Hausdorff} localmente compacto.
\end{thh}
\begin{thh}
	$\{B_\gamma (a)\colon r\in\Z, a\in\Qp\}$ es contable.
\end{thh}
\begin{thh}
	$\Qp$ es totalmente disconexo.
\end{thh}
	
	
\end{frame}
%20-----------------------------------------
\iffalse
\begin{frame}{Algunas definiciones de Topología}
\begin{df}
		Decimos que un espacio topológico es \tit{conexo} si no puede ser escrito como la unión de dos abiertos disyuntos no vacíos. Por otro lado, decimos que un espacio es \tit{disconexo} si es la unión de dos abiertos disyuntos no vacíos.
\end{df}

\begin{df}
Los subconjuntos conexos maximales de un espacio topológico son llamados \tit{componentes conexos}.
\end{df}

\begin{df}
	Decimos que un espacio topológico es \tit{totalmente disconexo} si todas sus componentes conexos son singletons.
\end{df}
	
\end{frame}
\fi
%21-----------------------------------------
\begin{frame}{Un corolario simple y bonito}
	\begin{lm}
		Sean $x, y \in \bb{Q}_p$ tales que $\norm{x}_p \neq \norm{y}_p$, entonces:
		$$\norm{x+y}_p=\maxx\{\norm{x}_p,\norm{y}_p\}.$$
	\end{lm}
	\begin{crr}
		Todos los triángulos en $\Qp$ son isósceles.  
	\end{crr}
\end{frame}
%22-----------------------------------------
\begin{frame}{Un corolario simple y bonito}

\begin{figure}
	\centering
	\begin{tikzpicture}
	
	%   \draw [black!20]  (0,0) grid  (3,4);
	\draw[orange]  (0,0)-- (3,0)--  (1.5,4)--cycle;
	
	
	\draw  (0,0) circle  (0pt) node[anchor=north] {$y$};
	\draw  (3,0) circle  (0pt) node[anchor=north] {$z$};
	\draw  (1.5,4.5) circle  (0pt) node[anchor=north] {$x$};
	\draw  (3.5,2) circle  (0pt) node[anchor=north] {$\pnorm{x-z}$};
	\draw  (-0.5,2) circle  (0pt) node[anchor=north] {$\pnorm{x-y}$};
	\draw  (1.5,-0.2) circle  (0pt) node[anchor=north] {$\pnorm{z-y}$};  
	\end{tikzpicture}
	\caption{Todos los triángulos en $\bb{Q}_p$ son isósceles}
	\label{fig:1}
\end{figure}
\end{frame}
%24-----------------------------------------

%25-----------------------------------------
\iffalse
\begin{frame}{Propiedades de $\rho$}
	\begin{itemize}[<+- | alert@+>]
		\item {$\rho$ es una función continua, sobreyectiva, pero no inyectiva.}
		\item $|\rho (x)-\rho (y)| \leq\pnorm{x-y}, \text{ para todo } x, y \in \mathbb{Q}_{p}$.
		Es decir, $\rho$ satisface la desigualdad de Hölder,
		\item $\rho\left (p^{\gamma} x\right)=p^{-\gamma} \rho (x), \text{ para todo }x \in \mathbb{Q}_{p}.$
	\end{itemize}
\end{frame}
\fi
%------------------------Arithmetic------------------------
\subsection{Aritmética $p$-ádica}
\section*{Aritmética $p$-ádica}
\begin{frame}{Expansiones $p$-ádicas de enteros}
	\begin{itemize}[<+- | alert@+>]
		\item Podemos expandir por cualquier base $p$ un número $n\in\Z$
		\item El procedimiento es algorítmico:
		$$a_0 =n\bmod p\hspace{5mm} \Longrightarrow \hspace{5mm} n_1 = \frac{n-a_0}{p},$$ $$a_1 =n_1\bmod p \hspace{5mm} \Longrightarrow \hspace{5mm} n_2 = \frac{n_1-a_1}{p},$$ $$a_2 =n_2\bmod p \hspace{5mm} \Longrightarrow \hspace{5mm} n_3 = \frac{n_2-a_2}{p},$$
		$$\vdots$$
		\item Así, la representación de un entero $p$-ádico por dígitos está dada por \ref{notacion}
		$$n={ a_{l} \ldots a_{3} a_{2} a_{1} a_{0}}_p,$$
		\item La representación es conocida como el \textit{Código de Hensel} de $n$.
	\end{itemize}
\end{frame}

\begin{frame}{Expansiones $p$-ádicas de enteros}
\begin{exampleblock}{Ejemplo}
	Sea $n=5353$ y sea $p=5$, entonces la representación $p$-ádica de 5353 en base 5 está dada por:
	\begin{align*}
	%\centering
	a_0 =5353\bmod5 = 3 \hspace{3mm} \Longrightarrow \hspace{3mm} n_1 = \frac{5353-3}{5}=1070,\\
	a_1 =1070\bmod 5=0 \hspace{3mm} \Longrightarrow \hspace{5mm}n_2 = \frac{1070-0}{5}=214,\\
	a_2 =214\bmod 5=4 \hspace{5mm} \Longrightarrow \hspace{9mm}n_3 = \frac{214-4}{5}=42,\\
	a_3 =42 \bmod 5=2 \hspace{7mm} \Longrightarrow \hspace{13mm}n_4 = \frac{42-2}{5}=8,\\
	a_4 =8 \bmod 5=3 \hspace{9mm} \Longrightarrow \hspace{15mm}n_5 = \frac{8-3}{5}=1,\\
	a_5 =1 \bmod 5=1 \hspace{9mm} \Longrightarrow \hspace{15mm}n_6 = \frac{1-1}{5}=0.
	\end{align*}
	En otras palabras, el código de Hensel de $5353$ es $132403_5$.
\end{exampleblock}
\end{frame}

\begin{frame}{Expansiones $p$-ádicas de racionales}
Consideremos $x$ tal que su serie de expansión es
$$
\begin{aligned}
x &=2+3 p+p^{2}+3 p^{3}+p^{4}+3 p^{5}+p^{6}+\cdots \\
&=2+3 p\left (1+p^{2}+p^{4}+\cdots\right)+p^{2}\left (1+p^{2}+p^{4}+\cdots\right) \\
&=2+\left (3 p+p^{2}\right)\left (1+p^{2}+p^{4}+\cdots\right).
\end{aligned}
$$
Como $1+p^{2}+p^{4}+\cdots$ converge a $\left (1-p^{2}\right)^{-1},$ tenemos
$$
x=2+\frac{3 p+p^{2}}{1-p^{2}}.
$$ 
Como caso particular, tomando $p=5$, tenemos que 
$$
x=2+\frac{3\cdot5+5^{2}}{1-5^{2}}=\frac{1}{3},
$$ 
por lo tanto, la expansión $5$-ádica de $\frac{1}{3}$ es $\cdots1313132_5$.
\end{frame}

\begin{frame}{Ejemplos de expansiones $p$-ádicas sobre racionales}
\begin{exampleblock}{Ejemplo}
		\begin{align*}
	&14.31_5=1 \cdot 5^{-2}+3 \cdot 5^{-1}+4 \cdot 5^{0}+1 \cdot 5^{1}=241 / 25 \\
	&1413_5=1 \cdot 5^{0}+3 \cdot 5^{1}+4 \cdot 5^{2}+1 \cdot 5^{3} =241 \\
	&14310_5=0 \cdot 5^{0}+1 \cdot 5^{1}+3 \cdot 5^{2}+4 \cdot 5^{3}+1 \cdot 5^{4}=1205
	\end{align*}
\end{exampleblock}
\end{frame}

\begin{frame}{Suma}
Sean $\alpha= (a_i)$ y $\beta = (b_i)$ dos enteros $p$-ádicos. Definimos la suma como una sucesión $ (c_i)$ de dígitos $p$-ádicos apoyados de una sucesión $ (\epsilon_i)$ en $\{ 0 , 1\}$  (\tit{carries}), tales que:
\begin{itemize}[<+- | alert@+>]
	\item $\epsilon_0=0,$
	\item $c_i=a_i + b_i + \epsilon_i $ ó $c_i=a_i + b_i + \epsilon_i - p$, donde alguno de los dos es un dígito $p$-ádico; es decir, $c_i\in \{0, \dots , p-1\}$. Dado el caso de $c_i$ se tendrá que $\epsilon_{i+1}=0$ o  $\epsilon_{i+1}=1.$
\end{itemize}
\end{frame}

\begin{frame}{Suma}

\begin{exampleblock}{Ejemplo}
	\begin{itemize}
	\item Tomando $p=7$, se tiene: 
	$$
	\vbox{
		\openup2pt
		\def\trule{\noalign{\smallskip\hrule\smallskip}}
		\halign{&\tabskip1em$\mathstrut####$\cr
			& \cdots  & 2 & 5 & 1 & 4 & 1 & 3\cr
			+     & \cdots  & 1 & 2 & 1 & 1 & 0 & 2 \cr
			\trule
			& \cdots  & 4 & 0 & 2 & 5 & 1 & 5 \cr
		}
	}
	%\bye
	$$
	
	\item $0-1$ en los $7$-ádicos: 
	$$
	\vbox{
		\openup2pt
		\def\trule{\noalign{\smallskip\hrule\smallskip}}
		\halign{&\tabskip1em$\mathstrut####$\cr
			& \cdots  & 0 & 0 & 0 & 0 & 0 & 0\cr
			-     & \cdots  & 0 & 0 & 0 & 0 & 0 & 1 \cr
			\trule
			& \cdots  & 6 & 6 & 6 & 6 & 6 & 6 \cr
		}
	}
	%\bye
	$$
	Esto quiere decir que $-1= \cdots 666_7$.
		
	\end{itemize}
	
\end{exampleblock}

\end{frame}

\iffalse
\begin{frame}{Representación de números negativos}
	Si $x=\sum_{i=\gamma}^{\infty}a_ip^i$, entonces $-x=\sum_{i=\gamma}^{\infty}b_ip^i$, donde $b_\gamma = p-a_\gamma$ y\linebreak	 $b_i =  (p-1)-a_i$ con $i>\gamma$.
	
	
	
	\begin{exampleblock}{Ejemplo} Con $p=5$
		\begin{align*}
		&\frac{1}{3}=\cdots1313132_5 \Rightarrow -\frac{1}{3}=\cdots3131313_5,\\
		&\frac{5}{3}=\cdots13131320_5 \Rightarrow -\frac{5}{3}=\cdots31313130_5.
		\end{align*}
	\end{exampleblock}
\end{frame}
\fi

\begin{frame}{Numeros unidades}
	\begin{df}
		Un número $p$-ádico es llamado \textit{unidad} si no es múltiplo de una potencia negativa de $p$ y su primer dígito no es $0$; más aún, un número $p$-ádico $x$ se puede escribir como:
		$${x=u\cdot p^{-N}},\text{ con $u$ unidad}.$$	
	\end{df}
	\begin{exampleblock}{Ejemplo}
		\begin{itemize}[<+- | alert@+>]
			\item Los números $\cdots314_5$ y $\cdots24_5$ son unidades.
			\item  $\cdots310_5$ y $\cdots1321.24_5$ no son unidades.
			\item Pero $\cdots 1321.24_5 = \cdots 132124_5\cdot 5^{-2}.$
		\end{itemize}

	\end{exampleblock}


\end{frame}

\begin{frame}{Multiplicación $p$-ádica}
	Sean $x=u\cdot p^{-N_1}$ y $y=v\cdot p^{-N_2}$ con $u,v$ unidades. Definimos la multiplicación $x\cdot y = u\cdot v\cdot p^{- (N_1+N_2)}$
	\begin{exampleblock}{Ejemplo}%\textcolor{red}{corregir este ejemplo, mejor usar la notacion del koblitz y que tengan potencias  negativas}
		\label{mult_ex}
		%Con $p=7$, sea $u = \cdots251413_7$ y $v=\cdots 121102_7$ números unidades. Luego
		%$$
		%\vbox{
		%	\openup2pt
		%	\def\trule{\noalign{\smallskip\hrule\smallskip}}
		%	\halign{&\tabskip1em$\mathstrut####$\cr
		%		& \cdots   & 5 & 1 & 4 & 1 & 3\cr
		%		\times     & \cdots   & 2 & 1 & 1 & 0 & 2\cr
		%		\trule
		%		& \cdots   & 3 & 3 & 1 & 2 & 6\cr
		%		& \cdots   & 0 & 0 & 0 & 0\cr
		%		& \cdots   & 4 & 1 & 3\cr
		%		& \cdots   & 1 & 3\cr
				%+ \cdots   & 6\cr
		%		+          & \cdots  & 6\cr 
		%		\trule
		%		& \cdots   & 1 & 0 & 4 & 2 & 6\cr
		%	}
		%}
		%\bye
		%$$
		
		%Así, con $p=7$, $u\cdot v =\cdots251413_7\times\cdots 123102_7=\cdots 310426_7$.
		
		\begin{center}
			\animategraphics[autoplay, every = 2, width=1.0\linewidth]{30}{Multiplicacion/multiplicacion00}{001}{179}
		\end{center}
	\end{exampleblock}
\end{frame}

\begin{frame}{División $p$-ádica}
	Los cálculos de divisiones en los enteros $p$-ádicos no difieren de los métodos tradicionales de división.
	\scalebox{0.9}{
	\vbox{
	\begin{exampleblock}{Ejemplo}
		%Tomando $p=7$, calculemos $\frac{\cdots421_7}{\cdots153_7}:$
		
		\begin{align*}
		\renewcommand\arraystretch{.75}\renewcommand\arraycolsep{3pt}
		\begin{array}{r@{\hskip\arraycolsep}c@{\hskip\arraycolsep}l*5r} % n=8=3+5
		&&5&1&6\cdots\\
		\cline{2-6} %n=8
		3\hspace{2mm} 5\hspace{2mm} 1&\Big)&1&2&4\cdots \\
		&&1&6&1\cdots \\
		\cline{3-6}\\
		&&&3&2\cdots\\
		&&&3&5\cdots\\
		\cline{4-6}\\
		&&&&4\cdots\\
		&&&&4\cdots\\
		\cline{5-6}\\
		&&&&&\cdots\\
		\end{array}
		\end{align*}
		Así, con $p=7$, $\frac{\cdots421_7}{\cdots153_7}=\cdots 615_7$.
	\end{exampleblock}
	}}
\end{frame}

\begin{frame}{División $p$-ádica}
	\scalebox{0.9}{
		\vbox{
\begin{rr}
	Los anteriores procedimientos de multiplicación y división, hechos sobre $\Zp$ pueden ser extendidos de manera natural a $\Qp$, pues el problema se reduce a operar números unidades.
\end{rr}

\begin{exampleblock}{Ejemplo}
	Al momento de multiplicar los números no-unidades, sean 
	\begin{align*}
	&x = \cdots2514.13_7= \cdots251413_7\cdot 7^{-2}=u\cdot 7^{-2},\\
	&y=\cdots 121.102_7=\cdots 121102_7\cdot 7^{-3}=v\cdot 7^{-3},
	\end{align*}
	Luego $$x\cdot y = u\cdot v \cdot 7^{- (2+3)}.$$
	Por el ejemplo \ref{mult_ex}, tenemos que $u\cdot v = \cdots 310426_7$, entonces:
	$$x\cdot y = \cdots 310426_7\cdot 7^{-5}=3.10426_7.$$

	
\end{exampleblock}
}}
\end{frame}
\subsection{Sucesiones y series de  números $p$-ádicos}
\section*{Sucesiones y series de  números $p$-ádicos}

\begin{frame}{El sueño del estudiante de cálculo}
\begin{thh}
	Si $$\lim_{n\to\infty}x_n=x, \text{ con } x_n,x\in\Qp\text{ y } \pnorm{x}\neq0 ,$$
	entonces la sucesión $ (\pnorm{x_n})_{n\in\N}$ se estabiliza, es decir, existe $N\in\N$ tal que:
	$$\pnorm{x_n}=\pnorm{x}, \text{ para todo } n\geq N.$$
\end{thh}
\begin{thh}
	Una serie $\sum_{j=1}^\infty x_j, \text{   } x_j\in\Qp$ converge en $\Qp$, si, y sólo si,  $\lim_{n\to\infty}x_n=0$. %En tal caso:
	%$$\pnorm{\sum_{j=1}^\infty x_j}\leq \maxx_{j}\pnorm{x_j}.$$
\end{thh}
\end{frame}

\begin{frame}{Ejemplos de series}
\begin{exampleblock}{Ejemplo}
	En $\bb{Q}_p$ tenemos que: $$	\sum_{n=1}^{\infty} n^{2}  (n+1) !=2.$$ 
\end{exampleblock}
\begin{block}{Problema abierto}
	Desde el año 1971 se abrió el siguiente problema:
	¿Puede ser $\sum_{n=0}^{\infty}n!$ un número racional para algún primo $p$?
	Por ahora, se sabe que $\sum_{n=0}^{\infty}n!$ converge en cada $\Qp$. Pero nada se sabe de su valor.
\end{block}
\end{frame}

\begin{frame}{Unicidad de la representación}
	\begin{pr}
		Todo número $p$-ádico se puede escribir de manera única como la suma de una serie convergente en  $\Qp$ de la forma:
		\begin{equation}
		\sum_{k=-\infty}^{\infty} a_{k} p^{k}\text{, con $a_k\in\{0,\dots,p-1\}$}	\label{eq:1}
		\end{equation}
		y en donde $a_k=0$, para $k\leq - N$ y $a_{-N}\neq 0$\label{ord_def_2}. A $-N$ se le denomina el \textit{orden} del número.
	\end{pr}
\end{frame}

\begin{frame}{Relación de $\Qp$ con $\R$}
	\begin{itemize}
		\item Existe una correspondencia con los \textit{Conjuntos de Cantor}.
		\item También podemos relacionarlos mediante una función $\rho\colon\Qp\to\R_{+}$ conocida como \textit{Monna map}, definida por
		\begin{equation}\label{Monna}
		\rho: \sum_{j=\gamma}^{\infty} a_{j} p^{j} \mapsto \sum_{j=\gamma}^{\infty} a_{j} p^{-j-1}, \quad a_{j}=0,1, \ldots, p-1, \quad \gamma \in \mathbb{Z}.
		\end{equation}
	\end{itemize}
\end{frame}
\iffalse
\begin{frame}{Parte entera y parte fraccionaria}
	\begin{itemize}[<+- | alert@+>]
		
		\item La \textit{\ parte fraccionaria de }$x\in\Qp$, denotada como
		$\left\{  x\right\}  _{p}$, es el siguiente número racional:
		\[
		\left\{  x\right\}  _{p}:=\left\{
		\begin{array}
		[c]{llll}%
		0 & \text{si} & x=0\text{,}  \text{ u}\,  \, \ord (x)\geq0\\
		p^{v}
		{\displaystyle\sum\limits_{j=0}^{\left\vert v\right\vert -1}}
		
		x_{j}p^{j} & \text{si} & \ord (x)<0. &
		\end{array}
		\right.
		\]
		\item Así, para todo $x\in\mathbb{Q}_{p}$ 
		\begin{align*}
		x & =\sum_{i=v}^{-1}a_{i}p^{i}+\sum_{i=0}^{\infty}a_{i}p^{i}\\
		& =:\left\{ x\right\} _{p}+[x]_{p}.
		\end{align*}
	\end{itemize}

\end{frame}
\fi

\subsection{Una mirada algebraica de los números $p$-ádicos}
\section*{Una mirada algebraica de los números $p$-ádicos}

\begin{frame}{Los enteros $p$-ádicos}
	\begin{df}
		El conjunto
		\[
		\mathbb{Z}_{p} =\{x\in\Qp:|x|_{p}\leq1\}=\{x\in\Qp:x=\sum_{i=i_{0}}^{\infty} a_{i}p^{i},i_{0}\geq0\},
		\]
		es llamado el conjunto de los \textit{enteros $p$-ádicos}.
	\end{df}
\begin{thh}
	$\Zp$ es un subanillo de $\Qp$.
\end{thh}
\end{frame}

\begin{frame}{Números invertibles}
\begin{pr}
	Un entero $p$-ádico $x=\sum_{i=i_{0}}^{\infty}a_{i}p^{i},i_{0}\geq0$ es invertible en $\Zp$ si, y sólo si, $a_0\neq 0$.
\end{pr}
\begin{itemize}
	\item Así, el grupo de los números invertibles en $\Zp$ está dado por:
	$$
	\mathbb{Z}_{p}^{\times}=\left\{x \in \mathbb{Z}_{p}:\pnorm{x}=1\right\}=\Big\{x \in \mathbb{Z}_{p}: x=\sum_{k=0}^{\infty} x_{k} p^{k}, \quad x_{0} \neq 0\Big\},$$
	que es un grupo multiplicativo del anillo $\Zp$.
	\item Estos elementos son llamados \tit{unidades} de $\Qp$ ¡Tal como lo vimos en la sección de aritmética!
\end{itemize}
\begin{exampleblock}{Ejemplo}
	$1-p$ es invertible en $\Zp$, pues su inverso es $\sum_{k=0}^{\infty}p^k=\frac{1}{1-p}.$
\end{exampleblock}
\end{frame}

\begin{frame}{Ideales en $\Zp$}
	\begin{itemize}[<+- | alert@+>]
		\item El anillo $\Zp$ es un \textit{dominio de ideales principales}.
		\item Más exactamente, cualquier ideal de $\mathbb{Z}_{p}$ tiene la forma
		\[
		p^{m}\mathbb{Z}_{p}=\Big\{x\in\mathbb{Z}_{p}:x=\sum_{i\geq m}a_{i}p^{i}%
		\Big\},\ m\in\mathbb{N}.
		\]
		\item $\mathbb{Z}_{p} \supset p \mathbb{Z}_{p} \cdots \supset p^{k} \mathbb{Z}_{p} \supset \cdots \supset \bigcap_{k \geq 0} p^{k} \mathbb{Z}_{p}=\{0\}$
		\item $\bb{Z}_p$ es un \tit{anillo local}, cuyo ideal maximal es:
		$$p\bb{Z}_p = \{x\in\bb{Z}_p : \pnorm{x}<1\}.$$
	\end{itemize}
\end{frame}
\iffalse
\begin{frame}{Homomorfismos}
	\begin{itemize}[<+- | alert@+>]
		\item Podemos definir el homomorfismo de anillos: \begin{align*}
		\pi_n: \Zp &\hspace{2mm}\rightarrow \Z/p^n\Z \\
		& x \mapsto  \sum_{k=0}^{n-1}a_{k}p^k\pmod{p^n},
		\end{align*}
		\item Y en general, definimos el homomorfismo:	\begin{align*}
		\pi:& \Zp \rightarrow \prod_{n=0}^{\infty}\Z/p^n\Z \\
		& x \hspace{2mm}\mapsto  (\pi_1 (x),\pi_2 (x),\dots) .
		\end{align*}
		\item Si nos restringimos a la imagen de este homorfismo, esta es conocida como el \tit{límite proyectivo} de los  $\Z/p^n\Z$, y se denota por $$\varprojlim_n \Z/p^n\Z$$ 
\end{itemize}
\end{frame}
\fi

\begin{frame}{Construcción de $\Zp$ y $\Qp$ vía álgebra}
	\begin{itemize}[<+- | alert@+>]
		\item $\Zp\cong\varprojlim_n \Z/p^n\Z.$
		\item $\bb{Q}_p=\operatorname{Frac} (\Zp).$
		\item $\bb{Q}_p=\bb{Z}_p[\frac{1}{p}]$.
	\end{itemize}
\end{frame}
\iffalse
\section{Sobre Diferenciación e Integración}

\begin{frame}{Derivadas y primitivas}
	
		Si $f\colon\Qp\to\C$, estaríamos tentados a definir: \red{$$f^{\prime} (a)=\lim\limits_{h\to 0}\frac{f (a+h)-f (a)}{h}.$$}
	\begin{df}
		Una función $f\colon B_\gamma\subseteq\Qp\to\Qp$ se dice \tit{analítica} si:
		$$f (x)=\sum_{n=0}^{\infty}a_nx^n,$$
		con $x\in B_\gamma$, $a_n\in\Qp$. 
	\end{df}
	
\end{frame}
\begin{frame}{Derivadas y primitivas}
	\begin{df} Si $f\colon B_\gamma\subseteq\Qp\to\Qp$ es analítica, definimos
		\[
		f^{ (m)} (x)=\sum_{n=m}^{\infty} n (n-1) \cdots (n-m+1) a_{n} x^{n-m},
		\]
		\[
		\quad f^{ (-m)} (x)=\sum_{n=0}^{\infty} \frac{1}{ (n+1) (n+2) \cdots (n+m)} a_{n} x^{n+m},
		\] 
		como \textit{derivadas} y \textit{primitivas}, respectivamente.
	\end{df}
\end{frame}

\begin{frame}{Integración}
	Dado que $ (\Qp, +)$ es un grupo topológico localmente compacto, un resultado conocido en teoría de la medida establece que $ (\Qp, +)$ tiene
	una única medida $dx$, llamada la \textit{medida de Haar}	de $\Qp$.
	\begin{df}
		Decimos que una función $f\colon\Qp\to\C$  es \textit{integrable} en $\Qp$ si existe 
		$$\lim _{N \rightarrow \infty} \int_{B_{N}} f (x) dx.$$
		Por notación, decimos que $f\in\mathcal{L}^1(\Qp)$.
		
		
	\end{df}
\end{frame}
\fi

\section{Modelando números $p$-ádicos}
\begin{frame}{La clase Número}
	\begin{itemize}[<+- | alert@+>]
		\item La representación estándar de un número en $\Qp$ está dada por:
		\begin{equation}
		\sum_{k=-\infty}^{\infty} a_{k} p^{k}\text{, con $a_k\in\{0,\dots,p-1\}$ y $a_k=0$, para $k\leq - N$. 
		}	\label{num_rep}
		\end{equation}
		\item Para modelar estas representaciones, se hace necesario restringirnos al caso de sumas finitas.
		\item Las principales características del número las representamos por medio de \tit{atributos} y \tit{métodos}.
	\end{itemize}
\end{frame}


\begin{frame}{Diseño}

\begin{longtable}[c]{| c | c | c |}
	\caption{clase Número    (\textit{Number}).\label{number_class}}\\
	
	\hline
	\multicolumn{3}{| c |}{Number}\\
	\hline
	\textbf{Atributo} & \textbf{Tipo} & \textbf{Descripción}\\
	$p$ & integer & Número primo\\
	$n$ & integer & Número entero negativo tal que ${p^n\leq \pnorm{x}}$\\
	$N$ & integer & Número entero positivo tal que ${\pnorm{x}\leq p^N}$\\
	%$m    (privado)$ & integer & Índice positivo donde la suma empezará\\
	%$M    (privado)$ & integer & Índice positivo donde la suma terminará\\
	\hline
	\textbf{Método} & \textbf{Retorno} & \textbf{Función}\\
	show   ()& void & Muestra los dígitos del número\\
	order   () & integer & Retorna el orden del número. Ver \ref{ord_def_1} y \ref{ord_def_2}.\\
	
	len   () & integer & Retorna la cantidad de dígitos del número\\
	norm   () & float & Calcula la norma del número. Ver  \ref{ord_def_1}\\
	
	\hline
\end{longtable}
Representación: $x  = {a_{-n} \cdots a_0 \cdots. a_{-N}}_p$\label{rep_num}
\end{frame}

\begin{frame}[fragile]{Ejemplos de uso en \textit{Python}}
	Sea $x=342.536_7=6\cdot7^{-3}+3\cdot7^{-2}+5\cdot7^{-1}+2\cdot7^0+4\cdot7+3\cdot7^2$

\begin{lstlisting}[language=Python, caption = Instancia de la clase Número   (Number), basicstyle=\tiny]
digits = [3,4,2,5,3,6]
x = Number(7,-3,2,digits) #initialization of x

x.show()
>> [3,4,2,5,3,6]

x.order()
>> -2

x.norm()
>> 49

x.len()
>>6
\end{lstlisting}
En este caso $p=7$, $n=-3$ y  $N=2$. Además satisface que $7^{-3}\leq\pnorm{x}\leq 7^2$.
\end{frame}

\begin{frame}{El conjunto $\gpn$}
	\begin{itemize}[<+- | alert@+>]
		\item \textcolor{orangeBar}{Problema}: $\Qp$ es infinito, y tiene elementos con posibles infinitos dígitos.
		\item \textcolor{lightGreen}{Alternativa}: Consideramos un subconjunto finito, el cual modelaremos.
		\item Lo modelamos bajo el \tit{paradigma orientado a objetos}
		\item Definimos a $\gpn$ como un subconjunto de $\Qp$ tal que:
		
		
		
		\begin{equation}\label{GpnN}
		\gpn := \Big\{x\in \Qp : x = \sum_{k=-N}^{-n} a_{k} p^{k} \text{ y } p^{n}\leq \pnorm{x}\leq p^N\Big \},
		\end{equation}
		
		donde $a_k\in \{0,\cdots,p-1\}$, $n\leq0$ y $N\geq 0$. La representación por dígitos está dada por \ref{rep_num}.
		
	\end{itemize}
\end{frame}

\begin{frame}{Observaciones de $\gpn$}
	\begin{itemize}[<+- | alert@+>]
		\item $   (\gpn,+)$ es un grupo abeliano de oden $p^{N-n+1}$.
		\item Esta noción de grupo aditivo está relacionada con la topología de bolas en $\Qp$
		\item Así, podemos pensar en $\gpn$ como el cociente de los grupos aditivos $B_n   (0)/B_N   (0)$
		\item  La  multiplicación y la división no son operaciones cerradas en $\gpn$
	\end{itemize}
\end{frame}

\begin{frame}{Diseño de la clase $\gpn$ - Atributos}
	
	\begin{longtable}[c]{| c | c | p{6cm} |}
		\caption{Atributos de la clase $\gpn$.\label{GmMp_class}}\\
		
		\hline
		\multicolumn{3}{| c |}{$\gpn$}\\
		\hline
		\textbf{Atributo} & \textbf{Tipo} & \textbf{Descripción}\\
		$p$ & integer & Número primo\\
		$n$ & integer & Número entero negativo tal que ${p^n\leq \pnorm{x}}$\\
		$N$ & integer & Número entero positivo tal que ${\pnorm{x}\leq p^N}$\\
		%$m   (privado)$ & integer & Índice positivo donde la suma empezará\\
		%$M   (privado)$ & integer & Índice positivo donde la suma terminará\\
		numbers & seq[Number] & Contenedor con números de la clase número   (\textit{Number})\\
		
		\hline

	\end{longtable}
\end{frame}

\begin{frame}{Diseño de la clase $\gpn$ - Métodos}
\scalebox{0.9}{
\vbox{
\begin{longtable}[c]{| c | c | p{5.5cm} |}

	\caption{Métodos principales de la Clase $\gpn$.}\\
	
	\hline
	\multicolumn{3}{| c |}{$\gpn$}\\
	
	
	\hline
	
	\textbf{Método} & \textbf{Retorno} & \textbf{Función}\\
	%GpnN\_export   ()& void & Crea un archivo con todos los posibles números y sus normas $p$-ádicas\\
	console\_printing   () & void & Imprime por consola los números  pertenecientes a este conjunto\\
	
	p\_sum   (n1, n2) & Number & Retorna la suma de dos números \\
	p\_sub   (n1, n2) & Number & Retorna la resta de dos números \\
	p\_mul   (n1, n2) & Number & Retorna el producto de dos números \\
	p\_div   (n1, n2) & Number & Retorna división de dos números \\
	p\_inverse   (n) & Number & Retorna el inverso multiplicativo de un número\\
	representation\_tree   () & void & Crea una imagen que representa  todo el conjunto\\
	\hline
\end{longtable}
}}

\end{frame}

\begin{frame}[fragile]{Ejemplos de uso de $\gpn$ en Python - generate\_numbers }
	Por notación $-{R}=\_R$. Tomamos la instancia de $\gpn$, donde ${n=-3=\_{3},}$ $N=3$ y $p=7$. Así:  
	\begin{equation}\label{finite_number}
	\mathit{G7\_{3}3} = \Big\{x\in \Qp : x = \sum_{k=-3}^{3} a_{k} 7^{k} \text{ y } 7^{-3}\leq\pnorm{x}\leq 7^3\Big\},
	\end{equation}
 con $a_k\in \{0,\cdots,6\}$. En \texttt{Python:}
 \begin{lstlisting}[language = Python, caption = Inicialización de la clase $\mathit{G7\_33}$,basicstyle=\tiny]
 G7_33 = GpnN(7,-3,3) #initialization
 G7_33.generate_numbers()
 \end{lstlisting}
 La segunda línea genera todos los posibles números de la forma $ \sum_{k=-3}^{3} a_{k} 7^{k}$ y los guarda en una lista de números de tipo Número   (Number).
\end{frame}

\begin{frame}[fragile]{Ejemplos de uso de $\gpn$ en Python - console\_printing }
	Si quisiéramos visualizar los números asociados a $\mathit{G7\_{3}3}$, podemos verlos listados usando el método \textit{console\_printing} de la clase $\gpn$:
	\begin{lstlisting}[language = Python, caption = Visualización de números en $\mathit{G7\_33}$,basicstyle=\tiny]
	G7_33.console_printing()
	
	>>[0,0,0,0,0,0,0]
	>>[0,0,0,0,0,0,1]
	.
	.
	.      
	>>[6,6,6,6,6,6,5]
	>>[6,6,6,6,6,6,6]
	\end{lstlisting}
	\begin{rr}
		El orden de los números en $\gpn$ coincide con el orden inducido en $\R_+$ a través de la función \ref{Monna} definida como \textit{Monna map}.
	\end{rr}
\end{frame}

\begin{frame}[fragile]{Ejemplos de uso de $\gpn$ en Python - Suma(p\_sum) }
Si quisieramos sumar $ x =4324.321_5$ y $y=23.4123_5$, notamos que $x,y$ son de la  forma $x=\sum_{-3}^{3}a_k5^k$ y $y=\sum_{-4}^{1}b_k5^k$; así, el mínimo $\gpn\subset\Q_5$ que contiene a $x,y$ es $G5\_{3}4$. Luego $x+y$ en \texttt{Python:}

\begin{lstlisting}[language = Python, caption = suma de números en $\mathit{G5\_34}$, basicstyle=\tiny]
G5_34 = GpnN(5,-3,4) #initialization

x_digits = [4,3,2,4,3,2,1,0]
y_digits = [0,0,2,3,4,1,2,3]
x = Number(5,-3,4,x_digits)
y = Number(5,-3,4,y_digits)

x_plus_y = G5_34.p_sum(x,y)
x_plus_y.show()
>>[4, 4, 0, 3, 2, 3, 3, 3]

\end{lstlisting}
Así $ 4324.321_5+23.4123_5=4403.2333_5$.
\end{frame}

\begin{frame}[fragile]{Ejemplos de uso de $\gpn$ en Python - Resta(p\_sub) }
Tomemos ${x =4324.321_5}$ y ${y=23.4123_5}$. De nuevo, construimos el mínimo $\gpn$ que contenga a $x,y$ y así  $\mathit{G5\_{3}4}$ será el subconjunto de $\Q_5$ que contiene a $x,y$.
\begin{lstlisting}[language = Python, caption = resta de números en $\mathit{G5\_34}$,basicstyle=\tiny]
G5_34 = GpnN(5,-3,4) #initialization

x_digits = [4,3,2,4,3,2,1,0]
y_digits = [0,0,2,3,4,1,2,3]

x = Number(5,-3,4,x_digits)
y = Number(5,-3,4,y_digits)

x_minus_y = G5_34.p_sub(x,y)
x_minus_y.show()
>>[4, 3, 0, 0, 4, 0, 3, 2]

\end{lstlisting}
Así $ 4324.321_5-23.4123_5=4300.4032_5$.
\end{frame}


\begin{frame}[fragile]{Ejemplos de uso de $\gpn$ en Python - Producto(p\_mul) }
	\begin{rr}
	\label{producto}  
	\begin{itemize}
		\item  Supongamos que se multiplica un número de $m$  dígitos con otro que tiene $n$  dígitos.
		\item El producto tiene a lo máximo $m+n$ dígitos.
		\item Realizar estos cómputos, representan costos   (computacionalmente) a medida que $m,n$ se hacen grandes   (cosa que no pasa con la suma).
		\item Nos restringimos a hacer multiplicaciones en el conjunto donde se esté trabajando.
	\end{itemize}	
	\end{rr}
\end{frame}


\begin{frame}[fragile]{Ejemplos de uso de $\gpn$ en Python - Producto(p\_mul) }
	\begin{itemize}
		\item Tomemos $x=3214.345356_7$ y $y = 53452.143_7$.
		\item El mínimo $\gpn\subset\Q_7$ que contiene a $x,y$ es $\mathit{G7\_{4}6}$.
		\item El producto tendrá $-n+N+1=-   (-4)+6+1=11$ dígitos a lo más, y no $10+8=18$ que es la suma de la cantidad de dígitos de $x,y$ respectivamente.
	\end{itemize}
	\begin{lstlisting}[language = Python, caption = producto de números en $\mathit{G7\_46}$,basicstyle=\tiny]
	G7_46 = GpnN(7,-4,6) #initialization
	x_digits = [0,3,2,1,4,3,4,5,3,5,6]
	y_digits = [5,3,4,5,2,1,4,3,0,0,0]
	
	x = Number(7,-4,6,x_digits)
	y = Number(7,-4,6,y_digits)
	
	x_by_y = G7_46.p_mul(x,y)
	x_by_y.show()
	
	>>[4, 0, 0, 6, 2, 0, 0, 5, 0, 0, 0]
	\end{lstlisting}
	Luego $3214.345356_7\cdot 53452.143_7=\cdots40062.005_7$
\end{frame}


\begin{frame}[fragile]{Ejemplos de uso de $\gpn$ en Python - División(p\_div) }
La observación \ref{producto}, hecha para el producto aplica para la división. Luego tomando $x,y$ como en el producto:

\begin{lstlisting}[language = Python, caption = división de números en $\mathit{G7\_46}$,basicstyle=\tiny]
G7_46 = GpnN(7,-4,6) #initialization

x_digits = [0,3,2,1,4,3,4,5,3,5,6]
y_digits = [5,3,4,5,2,1,4,3,0,0,0]

x = Number(7,-4,6,x_digits)
y = Number(7,-4,6,y_digits)

x_div_y = G7_46.p_div(x,y)
x_div_y.show()
>>[5, 1, 3, 3, 0, 3, 3, 2, 3, 6, 2]
\end{lstlisting}
Así $\frac{3214.345356_7}{53452.143_7}=\cdots51330.332362_7$
\end{frame}



\begin{frame}[fragile]{Árboles}
\begin{itemize}
	\item Denotamos por $M_n^m=\{x_1, \dots, x_{m^n}\}$ a las hojas del árbol con $n$ niveles y $m$ ramas.
	\item Podemos dotar a $M_n^m$, definimos la distancia en $M_n^m$ como $d   (x_i,x_j)=C^{N*}$ ($0<C<1$).
	\item Así, $   (M_n^m,d)$ es un espacio ultramétrico.
	
		\begin{figure}%
			\centering
			\subfloat[]{
				\centering
				\begin{tabular}{||c c c c||} 
					\hline
					$x_i$ & $x_j$ & $N*$ & distancia \\ [0.5ex] 
					\hline\hline
					$x_1$ & $x_2$ & 2 & $C^2$ \\ 
					\hline
					$x_1$ & $x_3$ &1 & $C$ \\
					\hline
					$x_1$ & $x_5$ & 0 & 1 \\ 
					\hline
				\end{tabular}
			}%
			%\qquad
			\subfloat[]{
				\centering
				\scalebox{0.55}{
				\vbox{
				\begin{tikzpicture}[level/.style={sibling distance=40mm/#1}]
				\node [circle,draw]    (z){}
				
				child {node [circle,draw]    (a) {}
					child {node [circle,draw]    (b) {}
						child {node {$x_1$}}
						child {node {$x_2$}}
					}
					child {node [circle,draw]    (g) {}
						child {node {$x_3$}}
						child {node {$x_4$}}
					}
				}
				child {node [circle,draw]    (j) {}
					child {node [circle,draw]    (k) {}
						child {node {$x_5$}}
						child {node {$x_6$}}
					}
					child {node [circle,draw]    (l) {}
						child {node {$x_7$}}
						child {node {$x_8$}}
					}
				};
				\end{tikzpicture}
			}}
			}%
			\caption{$M_3^2$}%
		\end{figure}
\end{itemize}	

\end{frame}


\begin{frame}[fragile]{Árboles-Python}
\begin{itemize}
	\item ¡$\gpn$ se puede representar a través de un árbol!
	\item Generalmente se toma $C=\frac{1}{p}$.
	\item Luego, un árbol finito puede representarse en \texttt{Python}, instanciando la función \texttt{representation\_tree()}:
	\begin{lstlisting}[language = Python, caption = Representación en árbol de $\mathit{G2\_21}$,basicstyle=\tiny]
	G2_21 = GpnN   (2,-2,1) #initialization
	G2_21.generate_numbers()
	G2_21.representation_tree()
	\end{lstlisting}
\end{itemize}
\end{frame}

\begin{frame}[fragile]{Árboles - Visualización}

\begin{figure}
	\captionsetup[subfigure]{font=footnotesize}
	\subcaptionbox{Números de la forma $\sum_{-1}^{2}a_k2^k$}[.42\textwidth]{%
		\includegraphics[scale=0.48]{tree_images/G2_21_dark.png}
	}%
	\subcaptionbox{Características}[.6\textwidth]{
		\scalebox{0.7}{
		\vbox{
		\begin{itemize}[<+- | alert@+>]
			
			\item Los números presentados satisfacen que $2^{-2}\leq\norm{x}_2\leq 2$.
			\item $a_k\in\{0,1\}$, con $k\in\{-1,0,1,2\}$.
			\item Decir que $a_k\in\{\textbf{\textcolor{orange}{0}},\textbf{\textcolor{red}{1}}\}$ será equivalente a decir que $a_k\in\{\textbf{\textcolor{orange}{Naranja}},\textbf{\textcolor{red}{Rojo}}\}$
			\item Si quisiéramos saber qué número es el que está ubicado en la parte derecha del árbol, de color \textbf{\textcolor{lightBlue}{azul}}, sólo seguimos el camino propuesto por la imagen.
			\item Desde la raíz: $\textbf{\textcolor{orange}{Naranja}}\to\textbf{\textcolor{orange}{Naranja}}\to\textbf{\textcolor{orange}{Naranja}}\to\textbf{\textcolor{red}{Rojo}}.$
			\item Es decir que el número es \textbf{\textcolor{orange}{000}},\textbf{\textcolor{red}{1}}$_2$ y tiene norma \textbf{\textcolor{lightBlue}{$2^{-1}$}}.
			\item El camino desde la raíz denota $a_2\to a_1\to a_0\to a_{-1}$.
		
		\end{itemize}
		}}
	}
\end{figure}
\end{frame}


\begin{frame}{Visualización de más $\gpn$'s}
	\begin{figure}
		\caption{Ejemplos de $2$-ádicos}
		\animategraphics[controls,width=0.6\linewidth]{15}{tree_images/}{1}{12}
	\end{figure}	
\end{frame}

\begin{frame}{Visualización de más $\gpn$'s}
	\begin{figure}
		\caption{Ejemplos de $5$-ádicos}
		\animategraphics[controls,width=0.6\linewidth]{15}{tree_images/G5}{4}{10}
	\end{figure}	
\end{frame}

\section{Laplaciano sobre árboles}
\begin{frame}{Matriz Laplaciana}
	\begin{df}
		Dado un grafo simple G con $n$ vértices, definimos la matriz Laplaciana $L\in \bb{M}_n(\R)$ como:
		$$ L = D-A,$$
		donde $D,A$ son las matrices de grados e incidencia del grafo, respectivamente.
	\end{df}
	Luego:
	$$
	L_{i, j}\coloneqq \left\{\begin{array}{ll}
	\operatorname{grado}\left(v_{i}\right) & \text { si } i=j, \\
	-1 & \text { si } i \neq j \text { y } v_{i} \text { es adyacente a } v_{j}, \\
	0 & \text { e.o.c. }
	\end{array}\right.$$
\end{frame}


\begin{frame}{Operador Laplaciano discreto}
\begin{itemize}[<+- | alert@+>]
	\item Supongamos una función $\vec{\phi}(t)$ que describe la distribución de calor dentro de un grafo en un tiempo dado.
	\item Así, $\phi_i(t)$ es el calor en el nodo $i$ en el tiempo $t$.
	\item El calor transferido entre dos nodos $i$ y $j$ es directamente proporcional a la diferencia de calor entre los mismos.
	\item Esto es, $\frac{d \phi_{i}(t)}{d t} =-k \sum_{j} A_{i j}\left(\phi_{i}(t)-\phi_{j}(t)\right)$.
	\item Ecuación que se puede llevar a la forma $\frac{d \vec{\phi}(t)}{d t}+k L \vec{\phi}(t)=0.$
	\item Esta ecuación tiene la forma de la ecuación de calor, salvo que $\nabla^2$ es $L$.
\end{itemize}
\end{frame}

\begin{frame}{Matriz de réplica}
 Si definimos $l\colon\{1,\dots,p^{N-n+1}\}\to GpnN \Rightarrow l(i)=x_i$,
 definimos la \tit{matriz de réplica} $\textbf{Q} $ de tamaño $p^{N-n+1}\times p^{N-n+1}$: $$Q_{ij}=\rho(\pnorm{l(i)-l(j)}),$$ 
 donde $\rho$ es una función que depende de la distancia $p$-ádica entre $l(i)$ y $l(j)$. Por ejemplo, con $p=2$ tenemos una matriz tipo \tit{Parisi}:
 \begin{center}	
 	\scalebox{0.8}{ 
 		$
 		\boldsymbol{Q}=\left(\begin{array}{lllllllll}
 		0 & q_{1} & q_{2} & q_{2} & q_{3} & q_{3} & q_{3} & q_{3} & \dots \\
 		q_{1} & 0 & q_{2} & q_{2} & q_{3} & q_{3} & q_{3} & q_{3} & \dots \\
 		q_{2} & q_{2} & 0 & q_{1} & q_{3} & q_{3} & q_{3} & q_{3} & \dots \\
 		q_{2} & q_{2} & q_{1} & 0 & q_{3} & q_{3} & q_{3} & q_{3} & \dots \\
 		q_{3} & q_{3} & q_{3} & q_{3} & 0 & q_{1} & q_{2} & q_{2} & \dots \\
 		q_{3} & q_{3} & q_{3} & q_{3} & q_{1} & 0 & q_{2} & q_{2} & \dots \\
 		q_{3} & q_{3} & q_{3} & q_{3} & q_{2} & q_{2} & 0 & q_{1} & \dots \\
 		q_{3} & q_{3} & q_{3} & q_{3} & q_{2} & q_{2} & q_{1} & 0 & \dots\\
 		&  &  &  & \vdots & & &  & \ddots
 		\end{array}\right)
 		$ 
 	}
 \end{center}
 

\end{frame}

\begin{frame}{Matriz de transición}
	
	Definida la matriz de réplica, introducimos la \tit{matriz de transición}:
	\[   
	W_{ij} =
	\begin{cases}
	Q_{ij} & i\neq j,\\
	-\sum_{\gamma \neq i}^{N-n+1}Q_{i\gamma} &i=j.\\
	
	\end{cases}
	\]
\begin{figure}
	\captionsetup[subfigure]{font=footnotesize}
	\centering
	\subcaptionbox{Matriz de transición de $\mathit{G2\_11}$}[.5\textwidth]{%
		\scalebox{0.7}{
			\centering
		$
		\mathcal{W}=\left(\begin{array}{llllllll}
		\textcolor{green3}{w_{0}} & \textcolor{green2}{w_{1}} & \textcolor{green1}{w_{2}} & \textcolor{green1}{w_{2}} & \textcolor{green0}{w_{3}} & \textcolor{green0}{w_{3}} & \textcolor{green0}{w_{3}} & \textcolor{green0}{w_{3}} \\
		\textcolor{green2}{w_{1}} & \textcolor{green3}{w_{0}} & \textcolor{green1}{w_{2}} & \textcolor{green1}{w_{2}} & \textcolor{green0}{w_{3}} & \textcolor{green0}{w_{3}} & \textcolor{green0}{w_{3}} & \textcolor{green0}{w_{3}} \\
		\textcolor{green1}{w_{2}} & \textcolor{green1}{w_{2}} & \textcolor{green3}{w_{0}} & \textcolor{green2}{w_{1}} & \textcolor{green0}{w_{3}} & \textcolor{green0}{w_{3}} & \textcolor{green0}{w_{3}} & \textcolor{green0}{w_{3}} \\
		\textcolor{green1}{w_{2}} & \textcolor{green1}{w_{2}} & \textcolor{green2}{w_{1}} & \textcolor{green3}{w_{0}} & \textcolor{green0}{w_{3}} & \textcolor{green0}{w_{3}} & \textcolor{green0}{w_{3}} & \textcolor{green0}{w_{3}} \\
		\textcolor{green0}{w_{3}} & \textcolor{green0}{w_{3}} & \textcolor{green0}{w_{3}} & \textcolor{green0}{w_{3}} & \textcolor{green3}{w_{0}} & \textcolor{green2}{w_{1}} & \textcolor{green1}{w_{2}} & \textcolor{green1}{w_{2}} \\
		\textcolor{green0}{w_{3}} & \textcolor{green0}{w_{3}} & \textcolor{green0}{w_{3}} & \textcolor{green0}{w_{3}} & \textcolor{green2}{w_{1}} & \textcolor{green3}{w_{0}} & \textcolor{green1}{w_{2}} & \textcolor{green1}{w_{2}} \\
		\textcolor{green0}{w_{3}} & \textcolor{green0}{w_{3}} & \textcolor{green0}{w_{3}} & \textcolor{green0}{w_{3}} & \textcolor{green1}{w_{2}} & \textcolor{green1}{w_{2}} & \textcolor{green3}{w_{0}} & \textcolor{green2}{w_{1}} \\
		\textcolor{green0}{w_{3}} & \textcolor{green0}{w_{3}} & \textcolor{green0}{w_{3}} & \textcolor{green0}{w_{3}} & \textcolor{green1}{w_{2}} & \textcolor{green1}{w_{2}} & \textcolor{green2}{w_{1}} & \textcolor{green3}{w_{0}}  \\
		\end{array}\right)
		$ 
	}
	}%
	\subcaptionbox{Árbol de representación de $\mathit{G2\_11}$}[.5\textwidth]{
		\begin{tikzpicture}
		[thick,scale=0.6, every node/.style={scale=0.6},
		every node/.style={fill=green0,circle, draw, inner sep=2pt},
		level 1/.style={sibling distance=40mm,nodes={fill=green1}},
		level 2/.style={sibling distance=20mm,nodes={fill=green2}},
		level 3/.style={sibling distance=10mm,nodes={fill=green3}}]
		\node [circle,draw] (z){$w_3$}
		
		child {node [circle,draw] (a) {$w_2$}
			child {node [circle,draw] (b) {$w_1$}
				child {node {$1$}}
				child {node {$2$}}
			}
			child {node [circle,draw] (g) {$w_1$}
				child {node {$3$}}
				child {node {$4$}}
			}
		}
		child {node [circle,draw] (j) {$w_2$}
			child {node [circle,draw] (k) {$w_1$}
				child {node {$5$}}
				child {node {$6$}}
			}
			child {node [circle,draw] (l) {$w_1$}
				child {node {$7$}}
				child {node {$8$}}
			}
		};
		\end{tikzpicture}
	}
\end{figure}
$Q_{ij}=\rho(\pnorm{l(i)-l(j)})= \frac{C}{\pnorm{l(i)-l(j)}^\alpha + 1},\text{ $l(i)-l(j)\in GpnN$}.$
\end{frame}

\begin{frame}{Ejemplos de matrices de transición}
	\begin{figure}
		
		\caption{Matrices de Parisi asociadas a $\mathit{G2\_11}$, $\mathit{G2\_12}$, $\mathit{G2\_22}$, $\mathit{G2\_32}$, $\mathit{G2\_33}$, $\mathit{G2\_43}$, respectivamente. Con $\alpha=2$ y $C=3$}
		\animategraphics[controls,width=0.6\linewidth]{15}{matrix/}{1}{6}
	\end{figure}
\end{frame}

\begin{frame}{Ecuación de ultradifusión}
Análogamente a como se estableció la ecuación de calor sobre grafos, definimos la \textit{ecuación maestra} como sigue:
\begin{equation}
\label{ME}
\frac{du_i(t)}{dt} = \sum_{j\neq i}W_{ji}u_j(t) - \sum_{j\neq i}W_{ij}u_i(t), 
\end{equation}

donde $u_i$ es la probabilidad de transición, es decir, la probabilidad de pasar del estado $i$ al estado $j$ en el tiempo $t$.
Esta ecuación se puede llevar a:$$\frac{d\vec{u}(t)}{dt} = W\vec{u}(t).$$
Cuya solución será: $$\vec{u}(t)=\sum_{i=1}^{p^{N-n+1}}c_i(t)v_i,\quad c_i(t)=c_i(0)e^{-\lambda_it}. $$


\end{frame}

\begin{frame}{Proceso de difusión en $\gpn$}
	\begin{figure}
		\caption{Comportamiento de la solución en distintos $t\in[0,10]$ para la condición inicial $u(0)$ en forma de campana de Gauss, en $\mathit{G2\_33}$.}
		\animategraphics[autoplay,loop, every = 1, width=0.6\linewidth]{25}{difusion_normal/}{0}{99}
	\end{figure}
\end{frame}
\begin{frame}{Proceso de difusión en $\gpn$}
	\begin{figure}
		\caption{Comportamiento de la solución en distintos $t\in[0,100]$ para la condición inicial $u(0)$ con entradas aleatorias en $\mathit{G3\_33}$.}
		\animategraphics[autoplay,loop, every = 1, width=0.6\linewidth]{25}{random/}{0}{99}
	\end{figure}
\end{frame}

\section{Sobre el uso del paquete}

\begin{frame}[fragile]{Configuración de ambiente}
	\begin{itemize}[<+- | alert@+>]
		\item Se supone \texttt{Python} instalado (en \texttt{Linux} viene por defecto).
		\item En \texttt{Windows}, se recomienda la instalación de una distribución (\texttt{Anaconda}, \texttt{Python(x,y)} o \texttt{WinPython}).
		\item Se recomienda instalar el sistema de gestión de paquetes \texttt{pip}.
		\item A través de \texttt{pip} podemos instalar las dependencias que usamos:
		\begin{lstlisting}[caption=Terminal o símbolo del sistema,basicstyle=\tiny]
		$ pip install numpy scipy pandas matplotlib pygraphviz pydot networkx	
		\end{lstlisting}
		
	\end{itemize}
\end{frame}

\begin{frame}[fragile]{Instalación de nuestro paquete (\texttt{py-adics})}
	\begin{itemize}[<+- | alert@+>]
		\item ¡Usted... sí, usted también puede usarlo!
		\item Se recomienda la instalación del sistema de control de versiones, \texttt{Git} (en \texttt{Linux} viene por defecto).
		\item Con \texttt{pip} podemos instalar el paquete a través del comando:
		\begin{lstlisting}[caption=Terminal o símbolo del sistema,basicstyle=\tiny]
		$ pip install git+https://github.com/ed4st/py-adics-package	
		\end{lstlisting}
		\item Para testear la instalación, cree un script de \texttt{Python}:
		\begin{lstlisting}[language = Python, caption = test del paquete, basicstyle=\tiny]
		from padics.Number import Number #our package
		
		x = Number(7,-3,2,[3,2,1,3,4])
		print(x.order(),x.norm())
		>> -2 49
		\end{lstlisting}
		\item ¡Felicitaciones, puede usar lo visto en este trabajo!
	\end{itemize}
\end{frame}

\include{bibliografia.bib}
\begin{frame}{Notas del paquete - Trabajo a futuro}
	\begin{itemize}[<+- | alert@+>]
		\item Hacer \texttt{unit testing} sobre los algoritmos.
		\item Mejorar el control de excepciones.
		\item Mejorar documentación de uso.
		\item Crear más modulos para poder modificar el código de manera estructural.
		\item Implementar más algoritmos, como los de expansiones $p$-ádicas, o los asociados al estudio de álgebra (como el \textit{Lema de Hensel}), para hacer criptografía.
		\item Implementar la versión del \textit{algoritmo de Karatsuba} en números $p$-ádicos.
		\item Para las personas que no tienen suficiente dominio de \texttt{Linux}, sería muy útil crear una interfaz gráfica del paquete.
		\item Aplicar la representación a modelos en \tit{Biología}, para el estudio de proteínas y genomas.
	\end{itemize}
\end{frame}

\section*{Gracias \blacksmiley{}}


\begin{frame}[allowframebreaks]
	\frametitle{Referencias} 
	\begin{thebibliography}{100} % 100 is a random guess of the total number of
		%references
		\bibitem{Boney96} Shelkovich V. M. Albeverio S., Khrennikov A. Yu., ``Theory of p-adic distributions: linear and nonlinear models." \emph{Cambridge University Press}, 2010.
		\bibitem{Boney96} Neal Koblitz.p-adic Numbers, p-adic Analysis, and Zeta-Functions. Springer-Verlag New York, 1984.
		\bibitem{Boney96} G. Parisi; N. Sourlas. P-adic numbers and replica symmetry breaking. The European Physical Journal B / Condensed Matter and Complex Systems, 14:535–542, 2000.
		\bibitem{Boney96}  V. S. Vladimirov, I. V. Volovich, and E. I. Zelenov.p−adic analysis and mathematical physics. Advanced Mathematics: Computations and Applications, pages 128–141, 1994.
		\bibitem{Boney96} V. S. Anashin. Uniformly distributed sequences in computer algebra or how to construct program generators of random numbers. Journal of Mathematical Sciences, 89:1355–1390, 1998.
		\bibitem{Boney96} A. V. Avetisov, A. H. Bikulov, and V. A. Osipov.p−adic description of characteristic relaxation in complex systems.J. Phys. A: Math. Gen.,36:4239–4246, 2003.
		\bibitem{Boney96} Bikulov A.H. Kozyrev S. V. Osipov V. A. Avetisov, V. A. On the ultrametricity of the fluctuation dynamic mobility of protein molecules. (Russian) Tr. Mat. Inst. Steklova, 265:75–81, 2009.
		\bibitem{Boney96} V. A. Avetisov, A. H. Bikulov, S. V. Kozyrev, and V. A. Osipov. p-adic models of ultrametric diffusion constrained by hierarchical energy landscapes. Journal of Physics A: Mathematical and General, 35(2):177,2002.
		
		\bibitem{Boney96} Svetlana Katok.p-adic analysis compared with real. Student Mathematical Library 037. American Mathematical Society, 2007.
		
		\bibitem{Boney96} Anatoly Kochubei. Pseudo-differential equations and stochastics overnon-Archimedean fields. Number 244. New York, 2001.
		\bibitem{Boney96} V. S. Vladimirov, I. V. Volovich, and E. I. Zelenov. p−adic analysis and mathematical physics. Advanced Mathematics: Computations and Applications, pages 128–141, 1994.
	\end{thebibliography}
\end{frame}


\end{document}