\documentclass{beamer}
\usepackage[utf8]{inputenc}
\usetheme[background=dark,titleformat = smallcaps , block = fill,numbering = fraction, progressbar = head, titleformat title= smallcaps]{metropolis}           % Use metropolis theme

\usepackage {extarrows}
\usepackage {tikz}
\usepackage[spanish]{babel}
\usepackage{graphicx}
\usepackage{amssymb}
%\usepackage{amsfonts}

\usepackage{enumerate}
\usepackage{amsmath}
\usepackage{amsthm}
\usepackage{xcolor}
%\usepackage{amsfonts,amssymb,amsthm}

\usepackage{url}
\usepackage{enumerate}
\usepackage{commath}
\usepackage{multicol}
\usepackage{mathtools}
\usepackage{scrextend}
\usepackage{hyperref}
\usepackage{cleveref}
\usepackage{longtable}
\usepackage{bbm}
\usepackage{siunitx}
\usepackage{listings}
\usepackage{xcolor}
\usepackage{subcaption}
\usepackage{epigraph}


\definecolor{codegreen}{rgb}{0,0.6,0}
\definecolor{codegray}{rgb}{0.5,0.5,0.5}
\definecolor{codepurple}{rgb}{0.58,0,0.82}
\definecolor{backcolour}{rgb}{0.95,0.95,0.92}

\lstdefinestyle{mystyle}{
	backgroundcolor=\color{backcolour},   
	commentstyle=\color{codegreen},
	keywordstyle=\color{blue},
	numberstyle=\tiny\color{codegray},
	stringstyle=\color{red},
	basicstyle=\ttfamily\footnotesize,
	breakatwhitespace=false,         
	breaklines=true,                 
	captionpos=b,                    
	keepspaces=true,                 
	numbers=left,                    
	numbersep=5pt,                  
	showspaces=false,                
	showstringspaces=false,
	showtabs=false,                  
	tabsize=2
}

\lstset{style=mystyle}
\lstset{language=Python}
\lstset{frame=lines}
\lstset{caption={Insert code directly in your document}}
\lstset{label={lst:code_direct}}
\lstset{basicstyle=\footnotesize}


\newcommand{\bb}[1]{\mathbb{#1}}

%\newtheorem{theorem}{Teorema}[section]
%\theoremstyle{plain}
\newtheorem{acknowledgement}[theorem]{Acknowledgement}
\newtheorem{algorithm}[theorem]{Algorithm}
\newtheorem{axiom}[theorem]{Axiom}
\newtheorem{case}[theorem]{Case}
\newtheorem{claim}{Claim}
\newtheorem{conclution}[theorem]{Conclusión}
\newtheorem{condition}[theorem]{Condition}
\newtheorem{conjecture}[theorem]{Conjecture}
%\newtheorem{corollary}[theorem]{Corolario}
\newtheorem{criterion}[theorem]{Criterion}
\theoremstyle{definition}
%\newtheorem*{df}{Definición}
%\newtheorem{definition}[theorem]{Definición}
%\newtheorem{example}[theorem]{Ejemplo}
\newtheorem{exercise}[theorem]{Exercise}
%\newtheorem{lemma}[theorem]{Lema}
\newtheorem{notation}[theorem]{Notation}
%\newtheorem{problem}[theorem]{Problem}
\newtheorem{proposition}[theorem]{Proposición}
\newtheorem{remark}[theorem]{Nota}
%\newtheorem{solution}[theorem]{Solución}
\newtheorem{summary}[theorem]{Summary}
\numberwithin{equation}{section}

\definecolor{defColor}{HTML}{3ED597}
\newcommand{\marine}[1]{\textcolor{defColor}{#1}}


\definecolor{thColor}{HTML}{FA7E0A}
\newcommand{\orangee}[1]{\textcolor{thColor}{#1}}

\definecolor{rkColor}{HTML}{F72121}
\newcommand{\redd}[1]{\textcolor{rkColor}{#1}}




\newtheorem{df}{\marine{Definición}}
\newtheorem{thh}{\orangee{Teorema}}
\newtheorem{pr}{\orangee{Proposición}}
\newtheorem{lm}{\orangee{Lema}}
\newtheorem{rr}{\redd{Observación}}


%\newtheorem{defn}[]{Definición}
%\newenvironment{definition}{\begin{defn}}{\end{defn}}
%\newtheorem{definition}{Definition}[section]
%\newtheorem*{remark}{Remark}
%%%%%%%%
\newcommand{\tit}[1]{\textit{#1}}
\newcommand{\bsym}{\mathbf}
\newcommand{\Mod}[1]{\ (\mathrm{mod}\ #1)}
%\newcommand{\blue}[1]{\textcolor{blue}{#1}}
\newcommand{\red}[1]{\textcolor{red}{#1}}
\renewcommand{\geq}{\geqslant}
\renewcommand{\leq}{\leqslant}
\newcommand{\Rplus}{\mathds{R}_{^{+}}}
\newcommand{\N}{\mathbb{N}}
\newcommand{\Z}{\mathbb{Z}}
\newcommand{\R}{\mathbb{R}}
\newcommand{\C}{\mathbb{C}}
\newcommand{\Q}{\mathbb{Q}}
\newcommand{\ssi}{\longleftrightarrow}
\newcommand{\ent}{\longrightarrow}
\newcommand{\Qp}{\mathbb{Q}_p}  
\newcommand{\Qpn}{\mathbb{Q}_p^n}
\newcommand{\Zpn}{\mathbb{Z}_p^n}
\newcommand{\Zp}{\mathbb{Z}_p}
\newcommand{\Zd}{\mathbb{Z}_2}
%\newcommand{\abs}[1]{\left\vert #1 \right\vert}
%\newcommand{\norm}[1]{\|#1\|}
\newcommand{\pnorm}[1]{\|#1\|_p}
\newcommand{\maxx}[1]{\text{m\'ax} #1}
\newcommand{\xbar}[1]{\hskip 1.4pt\overline{\hskip-1.2pt #1\hskip -.6pt}\hskip 1.2pt}
\newcommand{\rb}{\raisebox{-.35ex}}

\DeclareMathOperator{\s}{\mathbf{S}}
\DeclareMathOperator{\f}{\mathcal{F}}
\DeclareMathOperator{\A}{\mathbb{A}}
\DeclareMathOperator{\dist}{dist} % The distance.
\DeclareMathOperator{\d^n}{\dif^{\,n}}
%\DeclareMathOperator{\d}{\dif}
\DeclareMathOperator{\Real}{Re}
\DeclareMathOperator{\ord}{Ord}
\DeclareMathOperator{\Dom}{Dom}
\DeclareMathOperator{\vol}{vol}
\DeclareMathOperator{\gpn}{\mathit{{GpnN}}}
%%%%%%%%%
%%%%%%%%%%%%%%%%% dashed integrals %%%%%%%%%%%%%%%%%%%%%
\DeclareSymbolFont{eulargesymbols}{U}{zeuex}{m}{n}
\DeclareMathSymbol{\intop}{\mathop}{eulargesymbols}{"52}
\usepackage[toc,page]{appendix}
\renewcommand{\labelitemi}{$\circ$}


\title{Una introducción a los números $p$-ádicos, su aritmética y algunas simulaciones en \textit{Python}}
\subtitle{Trabajo de grado presentado para optar por el título de
Matemático}
\date{\today}


\author{\bf{Autor: }Edgar Baquero 
	\\ \bf{Supervisor: }Leonardo Chacón. PhD.}

\institute{Pontificia Universidad Javeriana,
Facultad de Ciencias\\
Departamento de Matemáticas}
\begin{document}
  \maketitle
  \section{Notación}
  \begin{frame}{Notación}
    En la escuela nos enseñaron a separar los números por unidades, decenas y centenas. Por ejemplo el número $437$ tiene $7$ unidades, $3$ decenas y $4$ centenas. Es decir que podemos representar $437$ como:
    $$437 = 7\cdot10^0+ 3\cdot 10^1 + 4\cdot 10^2,$$
    El número $543.89$ como:
    $$543.89 = 9\cdot10^{-2}+8\cdot10^{-1}+3\cdot10^0+4\cdot10^1+5\cdot10^2.$$
    Que también se puede denotar como $543.89_{10}$.
  \end{frame}

  \begin{frame}{Notación}
  	Así:
  	\begin{itemize}
  		\item Se puede expandir un número por cualquier base $q$.
 		\item Ejemplos conocidos de sistemas de numeración son el \tit{octal, hexadecimal y binario}, entre otros.
  	\end{itemize}
  \begin{exampleblock}{Ejemplo}
	Podemos representar el siguiente número:
	$$2\cdot8^{-2}+2\cdot8^{-1}+3\cdot8^0+4\cdot8^1+7\cdot8^2,$$
	por $743.22$ ($q=8$), o también $743.22_8$.
  \end{exampleblock}
	 
  \end{frame}

%3------------------
\begin{frame}{Notación}
	\begin{itemize}
		\item Particularmente, estamos interesados en expansiones sobre bases primas.
		\item Por ejemplo con $p=2$, el ¡Sistema binario!
		\item En general, una expansión de la forma.	$$x=\sum_{k=-\gamma}^{l}a_kp^k,\text{ con $\gamma\in\Z$, $a_k\in \{0,\dots,p-1\}$ },$$
		será
		\begin{equation}\label{notacion}
		{a_{l} \ldots a_{2} a_{1} a_{0},a_{-1}a_{-2}\cdots a_{-\gamma}}_p.
		\end{equation}
		\item Siendo así, ahora sí empecemos.
	\end{itemize}

\end{frame}
%4-------------------
\section{El campo de los números $p$-ádicos}
\begin{frame}{El campo de los números $p$-ádicos}
\begin{df}\label{pnorm}
	Sea $K$ un cuerpo. Una \textit{norma} en $K$ es una función
	$\abs{\cdot}\colon K \to \R_{\geq 0}$ tal que para todo $x,y\in$ $K$
	satisface las siguientes propiedades:
	
	\begin{enumerate}
		\item[$\diamond$] $\abs{x} \geq0$, $\abs{x}
		=0\Longleftrightarrow x=0,$
		
		\item[$\diamond$] $\abs{xy}  =\abs{x} \left\vert 
		y\right\vert ,$
		
		\item[$\diamond$] $\left\vert x+y\right\vert \leq\abs{x} +\left\vert y\right\vert$.
	\end{enumerate}
\end{df}
	Además, una norma $\abs{\cdot}$ en $K$ define una métrica natural dada por\linebreak ${d (x,y)=\abs{x-y}}$.
\end{frame}
%5-----------------------------------------
\begin{frame}{El campo de los números $p$-ádicos}
\begin{df}
	Dos normas $\abs{\cdot}_1$, $\abs{\cdot}_2$ sobre un cuerpo $K$ se dicen \tit{equivalentes }si inducen la misma topología sobre $K$, i.e., todo abierto con respecto a una  topología también  lo es con respecto a la otra. Por notación decimos que $\abs{\cdot}_1\sim\abs{\cdot}_2$.
\end{df}

\begin{pr}\label{quiv_power}
	Sea $K$ un cuerpo con dos normas $\abs{\cdot}_1$, $\abs{\cdot}_2$. Entonces \linebreak$\abs{\cdot}_1\sim\abs{\cdot}_2$ si, y sólo si, existe $c\in\bb{R}_{>0}$ tal que $\abs{\cdot}_1=\abs{\cdot}_2^c$.
\end{pr}
\end{frame}
%6----------------------
\begin{frame}{El campo de los números $p$-ádicos}
\begin{pr}
	[Equivalencia Lipschitz ]\label{lipeq}
	Sea $K$ un cuerpo con dos normas $\abs{\cdot}_1$, $\abs{\cdot}_2$. Entonces $\abs{\cdot}_1\sim\abs{\cdot}_2$ si, y sólo si existen constantes $k_1,k_2$ positivas tales que:
	$$k_1\abs{x}_1<\abs{x}_2< k_2\abs{x}_1,$$
	para todo $x\in K$.
\end{pr}
\begin{pr}
	\label{equiv_cauchy}
	Sea $K$ un cuerpo con dos normas $\abs{\cdot}_1$, $\abs{\cdot}_2$ tales que ${\abs{\cdot}_1\sim\abs{\cdot}_2}$, entonces una sucesión $ (x_n)$ es de Cauchy respecto a $\abs{\cdot}_1$ si, y sólo si es de Cauchy respecto a $\abs{\cdot}_2$.
\end{pr}
\end{frame}

%7----------------------
\begin{frame}{El campo de los números $p$-ádicos}
\begin{df}
	Una norma $\norm{\cdot}$ sobre un cuerpo $K$ se dice \textit{no-arquimediana o ultramétrica}, si la condición $ (3)$  (en la definición \ref{pnorm})  es reemplazada por
	\begin{equation}\label{condiultra}
	\|x+y\| \le \maxx\{\|x\|, \|y\|\}, \forall x,y\in K.
	\end{equation} 
\end{df}
\begin{rr}
	Dado que
	$$\|x+y\| \le \maxx\{\|x\|, \|y\|\} \leq \|x\| + \|y\|, \forall x,y\in \mathbb{Q},$$ 
	la condición \ref{condiultra} es también llamada \tit{desigualdad triangular  fuerte}.
\end{rr}
\end{frame}

%8----------------------
\begin{frame}{El campo de los números $p$-ádicos}
\begin{df} \label{ord_def_1}
	Fijemos un primo $p$, sea $x\in\mathbb{Q\smallsetminus}\left\{  0\right\}  $ expresado de forma única como $x=p^{v}\frac{a}{b}$, donde $v$ es un entero y
	$a$, $b$ son  primos relativos con $p$.
	Definimos la función $\pnorm{\cdot}$ de la siguiente manera:
	\[
	\| x\| _{p}=p^{-v},
	\]
	donde el entero $v=v\left (  x\right)  $ se denomina el orden $p$\textit{-ádico de} $x$ y
	será denotado por $\ord\left (  x\right)  $. Por definición $\|0\|_p=0$, y  $\ord (0)=+\infty $.
\end{df}
\end{frame}
%9----------------------
\begin{frame}{El campo de los números $p$-ádicos}
	\begin{exampleblock}{Ejemplo}
		Cálculo de la función $\pnorm{\cdot}$ para distintos $p$'s.
		\[
		\left| -\frac{66}{500}\right| _{p}=\left| -\frac{33}{250}\right| _{p}=\left| -\frac{3\cdot11}{2\cdot5^3}\right| _{p}=\begin{cases}
		\frac{33}{250}       &\,\,\, \text{ si } \,\,\,p=\infty;\\ 
		2                    &\,\,\, \text{ si }\,\,\, p=2;\\
		\frac{1}{3}          &\,\,\, \text{ si } \,\,\,p=3;\\  
		5^3                  &\,\,\, \text{ si }\,\,\, p=5;\\   
		1                    &\,\,\, \text{ si }\,\,\, p=7;\\    
		\frac{1}{11}         & \,\,\,\text{ si }\,\,\, p=11;\\     
		& \vdots\\
		1                   &\,\,\,\text{ otro caso}.
		\end{cases}
		\]
	\end{exampleblock}
\end{frame}
%10-----------------------------------------------
\begin{frame}{El campo de los números $p$-ádicos}
	Nuestros 3 mosqueteros:
	\begin{thh}
		[\tit{Fórmula Adélica del producto}]
		Sea $x\in \bb{Q}$ tal que $x\neq0$, entonces:
		$$\displaystyle\prod_{p}^\infty \norm{x}_p=1, \text{ con } \norm{x}_\infty=\abs{x} \text{ y $p$ primo}.$$
	\end{thh}
\begin{thh}
		$\| \cdot \|_p$ es una norma no arquimediana.	
\end{thh}
\begin{thh}
	[Ostrowski]	\label{ostrowsky} Cualquier norma no trivial sobre $\mathbb{Q}$ es equivalente al
	valor absoluto usual, o a una norma p-ádica $\| \cdot\| _{p}%
	$, para algún primo $p$.
\end{thh}
\end{frame}
%11-----------------------------------------------
\begin{frame}{El campo de los números $p$-ádicos}
	\begin{rr}
			Las normas $\norm{\cdot}_p$ y $\norm{\cdot}_q$ no son quivalentes si $p$ y $q$ son primos distintos. Por ejemplo, sea $p=5$ y $q=7$, la sucesión 
		$x_n=\big (\frac{5}{7}\big)^n$ se tiene que 
		$$\norm{x_n}_5=5^{-n}\rightarrow 0 \text{ y } \norm{x_n}_7=7^{n}\rightarrow \infty ,$$
		cuando $n \rightarrow \infty$.
		
		El valor absoluto usual sobre $\Q$ tampoco es equivalente a una norma $p$-ádica. Por ejemplo, considérese la sucesión $x_n= (\frac{1}{p})^n$, entonces
		$$\abs{x_n}=p^{-n}\to 0 \text{ y } \norm{x_n}_p = p^n \to \infty,$$
		cuando $n\to\infty$. Lo cual contradice la proposición \ref{equiv_cauchy}.
	\end{rr}
\end{frame}
%12-----------------------------------------------
\begin{frame}{El campo de los números $p$-ádicos}
	\begin{rr}
		Las normas $\norm{\cdot}_p$ y $\norm{\cdot}_q$ no son quivalentes si $p$ y $q$ son primos distintos. Por ejemplo, sea $p=5$ y $q=7$, la sucesión 
		$x_n=\big (\frac{5}{7}\big)^n$ se tiene que 
		$$\norm{x_n}_5=5^{-n}\rightarrow 0 \text{ y } \norm{x_n}_7=7^{n}\rightarrow \infty ,$$
		cuando $n \rightarrow \infty$.
		
		El valor absoluto usual sobre $\Q$ tampoco es equivalente a una norma $p$-ádica. Por ejemplo, considérese la sucesión $x_n= (\frac{1}{p})^n$, entonces
		$$\abs{x_n}=p^{-n}\to 0 \text{ y } \norm{x_n}_p = p^n \to \infty,$$
		cuando $n\to\infty$. Lo cual contradice la proposición \ref{equiv_cauchy}.
	\end{rr}
\end{frame}
%13-----------------------------------------------
\begin{frame}{El campo de los números $p$-ádicos}
\begin{thh}
	[Caracterización de sucesiones de Cauchy]\label{car}
	Una sucesión $ (x_n)_{n\in\N}$ en $\Q$ es de Cauchy, si, y sólo si:
	\begin{equation}\label{car_cau}
	\lim_{n\to\infty}\pnorm{x_{n+1}-x_n}=0.
	\end{equation}
\end{thh}
\begin{df}
	Sea $ (\bb{K}, \norm{\cdot})$ un cuerpo métrico. Sea ${\bb{K}^\N}$ el anillo de todas las sucesiones en $\bb{K}$. Definimos $\mathcal C,\mathcal N $ como el subanillo de todas las sucesiones de Cauchy y el subanillo de todas las sucesiones finalmente nulas, respectivamente.
\end{df}
\end{frame}
%14-----------------------------------------------
\begin{frame}{El campo de los números $p$-ádicos}
\begin{df}
	[Completación de un cuerpo métrico]Sea $ (\bb{K}, \norm{\cdot})$ un cuerpo métrico. Sean $\mathcal C,\mathcal N $ los subanillos de todas las sucesiones de Cauchy y de todas las sucesiones finalmente nulas, respectivamente.  Definimos el cociente de anillos $\hat{\bb{K}}\coloneqq \mathcal C/\mathcal N  $ como la completación de $\bb{K}$.
\end{df}
\begin{rr}
	La norma $\norm{\cdot}\colon{\hat{\bb{K}}}\to\R_+,$ sobre la completación de $\bb{K}$ está definida tal que para todo $ (x_n)+\mathcal{N} \in \hat{\bb{K}}$:
	$$\norm{ (x_n)+\mathcal{N}}=\lim_{n\to\infty}\norm{x_n}.$$
\end{rr}
\end{frame}
%15-----------------------------------------------
\begin{frame}{El campo de los números $p$-ádicos}
	El siguiente teorema es importante, pues caracteriza a $\Q$ como un cuerpo no completo.
	\begin{thh}
		$ (\Q, d (x,y)=\pnorm{x-y})$ y $ (\Q, d (x,y)=|x-y|)$ no son espacios completos.
	\end{thh}
\begin{exampleblock}{Ejemplo}
	Un procedimiento para construir una sucesión de Cauchy en $ (\Q, d (x,y)=\pnorm{x-y})$ podría hacerse, tomando $a\in\Q$ tal que:
	\begin{itemize}
		\item[$\diamond$] $a$ no es cuadrado en $\Q$
		\item[$\diamond$] $p \nmid a$
		\item[$\diamond$] $a$ es residuo cuadrático módulo $p$. i.e., $x^2 \equiv a \Mod{p^n}$ tiene solución.
	\end{itemize}
\end{exampleblock}
\end{frame}
%16-----------------------------------------------
\begin{frame}{El campo de los números $p$-ádicos}
	\begin{exampleblock}{Continuación$\dots$}
		Podemos hallar $a$ tal que sea cuadrado en $\Z$ y sumarle un múltplo de $p$; para así construir la sucesión como sigue:
		\begin{itemize}
			\item[$\diamond$] Tomamos $x_0$ solución de $x^2\equiv a \Mod{p}$
			\item[$\diamond$] Construimos a $x_1$ tal que $x_1 \equiv x_0 \Mod{p}$ y además ${x_1^2\equiv a \Mod{p^2}}$ 
			\item[$\diamond$] Recursivamente, construimos $x_n$ tal que:
			$$x_n \equiv x_{n-1} \Mod{p^n} \text{ y } {x_n^2\equiv a \Mod{p^{n+1}}}$$
		\end{itemize}
	Es de cauchy: $  \pnorm{x_{n+1}-x_n} = \pnorm{kp^n} \leq \pnorm{p^n}=p^{-n}\rightarrow0.$\linebreak
	No converge: $\pnorm{x_n^2-a}=\pnorm{sp^{n+1}}\leq \pnorm{p^{n+1}}\leq p^{- (n+1)}\to 0,$
	luego $x_n\to \sqrt{a}\notin\Q$.
	\end{exampleblock}
\end{frame}


\end{document}