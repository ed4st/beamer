\documentclass{beamer}
\usetheme[background=dark,titleformat = smallcaps , block = fill,numbering = fraction, progressbar = head, titleformat title= smallcaps]{metropolis}           % Use metropolis theme

\usepackage {extarrows}
\usepackage {tikz}
\usepackage[spanish]{babel}
\usepackage{graphicx}
\usepackage{amssymb}
%\usepackage{amsfonts}

\usepackage{enumerate}
\usepackage{amsmath}
\usepackage{amsthm}
\usepackage{xcolor}
%\usepackage{amsfonts,amssymb,amsthm}

\usepackage{url}
\usepackage{enumerate}
\usepackage{commath}
\usepackage{multicol}
\usepackage{mathtools}
\usepackage{scrextend}
\usepackage{hyperref}
\usepackage{cleveref}
\usepackage{longtable}
\usepackage{bbm}
\usepackage{siunitx}
\usepackage{listings}
\usepackage{xcolor}
\usepackage{subcaption}
\usepackage{epigraph}
\usepackage{docmute}


\definecolor{codegreen}{rgb}{0,0.6,0}
\definecolor{codegray}{rgb}{0.5,0.5,0.5}
\definecolor{codepurple}{rgb}{0.58,0,0.82}
\definecolor{backcolour}{rgb}{0.95,0.95,0.92}

\lstdefinestyle{mystyle}{
	backgroundcolor=\color{backcolour},   
	commentstyle=\color{codegreen},
	keywordstyle=\color{blue},
	numberstyle=\tiny\color{codegray},
	stringstyle=\color{red},
	basicstyle=\ttfamily\footnotesize,
	breakatwhitespace=false,         
	breaklines=true,                 
	captionpos=b,                    
	keepspaces=true,                 
	numbers=left,                    
	numbersep=5pt,                  
	showspaces=false,                
	showstringspaces=false,
	showtabs=false,                  
	tabsize=2
}

\lstset{style=mystyle}
\lstset{language=Python}
\lstset{frame=lines}
\lstset{caption={Insert code directly in your document}}
\lstset{label={lst:code_direct}}
\lstset{basicstyle=\footnotesize}


\newcommand{\bb}[1]{\mathbb{#1}}

%\newtheorem{theorem}{Teorema}[section]
%\theoremstyle{plain}
\newtheorem{acknowledgement}[theorem]{Acknowledgement}
\newtheorem{algorithm}[theorem]{Algorithm}
\newtheorem{axiom}[theorem]{Axiom}
\newtheorem{case}[theorem]{Case}
\newtheorem{claim}{Claim}
\newtheorem{conclution}[theorem]{Conclusión}
\newtheorem{condition}[theorem]{Condition}
\newtheorem{conjecture}[theorem]{Conjecture}
%\newtheorem{corollary}[theorem]{Corolario}
\newtheorem{criterion}[theorem]{Criterion}
\theoremstyle{definition}
%\newtheorem{definition}[theorem]{Definición}
%\newtheorem{example}[theorem]{Ejemplo}
\newtheorem{exercise}[theorem]{Exercise}
%\newtheorem{lemma}[theorem]{Lema}
\newtheorem{notation}[theorem]{Notation}
%\newtheorem{problem}[theorem]{Problem}
\newtheorem{proposition}[theorem]{Proposición}
\newtheorem{remark}[theorem]{Nota}
%\newtheorem{solution}[theorem]{Solución}
\newtheorem{summary}[theorem]{Summary}
\numberwithin{equation}{section}

%\newtheorem{definition}{Definition}[section]

%%%%%%%%
\newcommand{\tit}[1]{\textit{#1}}
\newcommand{\bsym}{\mathbf}
\newcommand{\Mod}[1]{\ (\mathrm{mod}\ #1)}
\newcommand{\blue}[1]{\textcolor{blue}{#1}}
\newcommand{\red}[1]{\textcolor{red}{#1}}
\renewcommand{\geq}{\geqslant}
\renewcommand{\leq}{\leqslant}
\newcommand{\Rplus}{\mathds{R}_{^{+}}}
\newcommand{\N}{\mathbb{N}}
\newcommand{\Z}{\mathbb{Z}}
\newcommand{\R}{\mathbb{R}}
\newcommand{\C}{\mathbb{C}}
\newcommand{\Q}{\mathbb{Q}}
\newcommand{\ssi}{\longleftrightarrow}
\newcommand{\ent}{\longrightarrow}
\newcommand{\Qp}{\mathbb{Q}_p}  
\newcommand{\Qpn}{\mathbb{Q}_p^n}
\newcommand{\Zpn}{\mathbb{Z}_p^n}
\newcommand{\Zp}{\mathbb{Z}_p}
\newcommand{\Zd}{\mathbb{Z}_2}
%\newcommand{\abs}[1]{\left\vert #1 \right\vert}
%\newcommand{\norm}[1]{\|#1\|}
\newcommand{\pnorm}[1]{\|#1\|_p}
\newcommand{\maxx}[1]{\text{m\'ax} #1}
\newcommand{\xbar}[1]{\hskip 1.4pt\overline{\hskip-1.2pt #1\hskip -.6pt}\hskip 1.2pt}
\newcommand{\rb}{\raisebox{-.35ex}}

\DeclareMathOperator{\s}{\mathbf{S}}
\DeclareMathOperator{\f}{\mathcal{F}}
\DeclareMathOperator{\A}{\mathbb{A}}
\DeclareMathOperator{\dist}{dist} % The distance.
\DeclareMathOperator{\d^n}{\dif^{\,n}}
%\DeclareMathOperator{\d}{\dif}
\DeclareMathOperator{\Real}{Re}
\DeclareMathOperator{\ord}{Ord}
\DeclareMathOperator{\Dom}{Dom}
\DeclareMathOperator{\vol}{vol}
\DeclareMathOperator{\gpn}{\mathit{{GpnN}}}
%%%%%%%%%
%%%%%%%%%%%%%%%%% dashed integrals %%%%%%%%%%%%%%%%%%%%%
\DeclareSymbolFont{eulargesymbols}{U}{zeuex}{m}{n}
\DeclareMathSymbol{\intop}{\mathop}{eulargesymbols}{"52}
\usepackage[toc,page]{appendix}
\renewcommand{\labelitemi}{$\circ$}


\title{Una introducción a los números $p$-ádicos, su aritmética y algunas simulaciones en \textit{Python}}
\subtitle{Trabajo de grado presentado para optar por el título de
Matemático}
\date{\today}


\author{\bf{Autor: }Edgar Baquero 
	\\ \bf{Supervisor: }Leonardo Chacón. PhD.}

\institute{Pontificia Universidad Javeriana,
Facultad de Ciencias\\
Departamento de Matemáticas}
\begin{document}
	\maketitle
  \section{Los números $p$-ádicos}
  \begin{frame}{Notación}
    En la escuela nos enseñaron a separar los números por unidades, decenas y centenas. Por ejemplo: 
  \end{frame}

  	%\chapter{Modelando números $p$-ádicos}
\label{chapter2}


%Dado que el interés principal de este trabajo está enfocado al uso computacional de los números $p$-ádicos y dado que existen paquetes ofrecidos por \textit{Mathematica} para hacer cómputos aritméticos de números $p$-ádicos, encontramos que estos cómputos en general son puntuales, es decir, son operaciones número a número; Luego, la representación gráfica de los mismos queda en segundo plano. Además, la última fecha de revisión de los algoritmos es del año 1996; razón por la cual requerimos de una forma amigable y rápida de realizar estos cómputos, por lo tanto se hace necesario el uso de un lenguaje moderno y con nuevas estructuras de datos que permitan la eficiencia espacio-temporal\footnote{La eficiencia basada en el concepto de complejidad algorítmica.} de algoritmos sobre números $p$-ádicos.\\

%Usamos Python\footnote{Se requiere el uso de Python 3.5 en adelante.}, dado que es un lenguaje orientado a objetos y por ende permite un mejor diseño del cuerpo de números $p$-ádicos, además, permite representaciones simples de grafos y redes; más específicamente, nos basamos un paquete abierto ofrecido por la comunidad de desarrolladores, llamado %\textbf{NetworkX}\footnote{Para una documentación detallada acerca del paquete visitar: %\url{https://networkx.github.io/documentation/latest/_downloads/networkx_reference.pdf}}, dedicado a la creación, manipulación, estudio de estructuras, funciones y dinámica de redes complejas.

%\newpage

\section{La clase Número}
Como se mostramos en el capítulo \ref{chapter1}, la representación estándar de un número en $\Qp$ está dada por:
\begin{equation}
\sum_{k=-\infty}^{\infty} a_{k} p^{k}\text{, con $a_k\in\{0,\dots,p-1\}$ y $a_k=0$, para $k\leq - N$. 
}	\label{num_rep}
\end{equation}

Para modelar estas representaciones, se hace necesario restringirnos al caso de sumas finitas, luego, las principales características del número las representamos por medio de atributos y métodos.

\subsection{Diseño}
Definimos la clase Número     (\textit{number}) como sigue:


\begin{longtable}[c]{| c | c | c |}
	\caption{clase Número    (\textit{number}).\label{number_class}}\\
	
	\hline
	\multicolumn{3}{| c |}{Number}\\
	\hline
	\textbf{Atributo} & \textbf{Tipo} & \textbf{Descripción}\\
	$p$ & integer & Número primo\\
	$n$ & integer & Número entero negativo tal que ${p^n\leq \pnorm{x}}$\\
	$N$ & integer & Número entero positivo tal que ${\pnorm{x}\leq p^N}$\\
	%$m    (privado)$ & integer & Índice positivo donde la suma empezará\\
	%$M    (privado)$ & integer & Índice positivo donde la suma terminará\\
	\hline
	\textbf{Método} & \textbf{Tipo-retorno} & \textbf{Función}\\
	show   ()& void & Muestra los dígitos del número\\
	order   () & integer & Retorna el orden del número. Ver \ref{ord_def_1} y \ref{ord_def_2}.\\
	
	len   () & integer & Retorna la cantidad de dígitos del número\\
	norm   () & float & Calcula la norma del número. Ver  \ref{ord_def_1}\\
	
	\hline
\end{longtable}

%Nótese que m y M son atributos privados ya que simplemente siguen la convención $m=N$ y $M=-n$. 
Así, la representación del número será de la forma:

\begin{equation}\label{rep_num}
x  = {a_{-n} \cdots a_0 \cdots a_{-N}}_p
\end{equation}

Para la implementación, ver por ejemplo \ref{ord_def_2}

\subsection{Ejemplos de uso}
Sea $x=342.536_7=6\cdot7^{-3}+3\cdot7^{-2}+5\cdot7^{-1}+2\cdot7^0+4\cdot7+3\cdot7^2$ y representémoslo como una instancia de la clase Número   (Number). En este caso $p=7$, $n=-3$ y  $N=2$. Además satisface que $p^{-3}\leq\pnorm{x}\leq p^2$. En Python:

\begin{lstlisting}[language=Python, caption = Instancia de la clase Número   (Number)]
digits = [3,4,2,5,3,6]
x = Number(7,-3,2,digits) #initialization of x

x.show()
>> [3,4,2,5,3,6]

x.order()
>> -2

x.norm()
>> 49

x.len()
>>6
\end{lstlisting}



\section{La clase $\gpn$ }

Dado que $\Qp$ es infinito, y tiene elementos con posibles infinitos dígitos, se hace necesario considerar un subconjunto finito, el cual modelaremos siguiendo un esquema de diseño orientado a objetos.
\subsection{$\gpn$}

Definimos a $\gpn$ como un subconjunto de $\Qp$ tal que:



\begin{equation}\label{GpnN}
\gpn := \Big\{x\in \Qp : x = \sum_{k=-N}^{-n} a_{k} p^{k} \text{, $a_k\in \{0,\cdots,p-1\}$ y } p^{n}\leq \pnorm{x}\leq p^N\Big \},
\end{equation}

donde $n\leq0$ y $N\geq 0$ y la representación por dígitos está dada por \ref{rep_num}.

Nótese que  $   (\gpn,+)$ es un grupo abeliano de oden $p^{N-n+1}$. Esta noción de grupo aditivo viene dada por el capítulo anterior, donde estudiábamos algunas propiedades topológicas y asociábamos una estructura natural de grupo sobre las bolas $\{B_\gamma   (0)\colon\gamma\in\Z\}$. Luego, podemos pensar en $\gpn$ como el cociente de los grupos aditivos $B_n   (0)/B_N   (0)$   (donde $N>0$ y $n\leq0$ son enteros). Sin embargo, nos gustaría tener un poco más de estructura, pero, la  multiplicación y la división no son operaciones cerradas en $\gpn$.

\subsection{Diseño}
Definimos la clase $\gpn$ como sigue:

\begin{longtable}[c]{| c | c | p{6cm} |}
	\caption{Clase $\gpn$.\label{GmMp_class}}\\
	
	\hline
	\multicolumn{3}{| c |}{$\gpn$}\\
	\hline
	\textbf{Atributo} & \textbf{Tipo} & \textbf{Descripción}\\
	$p$ & integer & Número primo\\
	$n$ & integer & Número entero negativo tal que ${p^n\leq \pnorm{x}}$\\
	$N$ & integer & Número entero positivo tal que ${\pnorm{x}\leq p^N}$\\
	%$m   (privado)$ & integer & Índice positivo donde la suma empezará\\
	%$M   (privado)$ & integer & Índice positivo donde la suma terminará\\
	numbers & seq[number] & Contenedor con números de la clase número   (\textit{Number})\\
	
	\hline
	
	\textbf{Método} & \textbf{Tipo-retorno} & \textbf{Función}\\
	GpnN\_export   ()& void & Crea un archivo con todos los posibles números y sus normas $p$-ádicas\\
	console\_printing   () & void & Imprime por consola los números pertenecientes a este conjunto\\
	
	p\_sum   (n1:number, n2:number) & number & Retorna la suma de dos números \\
	p\_sub   (n1:number, n2:number) & number & Retorna la resta de dos números \\
	p\_mul   (n1:number, n2:number) & number & Retorna el producto de dos números \\
	p\_div   (n1:number, n2:number) & number & Retorna división de dos números \\
	p\_inverse   (n:number) & number & Retorna el inverso multiplicativo de un número\\
	representation\_tree   () & void & Crea una imagen que representa todo el conjunto\\
	\hline
\end{longtable}
%\begin{remark}
%	De nuevo, recalcamos que $m$ y $M$ son atributos privados ya que simplemente siguen la convención $m=N$ y $M=-n$. Esto, con el fin de hacer más sencilla la programación. Ver \ref{impl2}.

%\end{remark}

\subsection{Ejemplos de uso}
Por notación $-{R}=\_R$. Tomamos la instancia de $\gpn$, donde ${n=-3=\_{3},}$ $N=3$ y $p=7$. Así:  
\begin{equation}\label{finite_number}
\mathit{G7\_{3}3} = \Big\{x\in \Qp : x = \sum_{k=-3}^{3} a_{k} 7^{k} \text{, con $a_k\in \{0,\cdots,6\}$ y } 7^{-3}\leq\pnorm{x}\leq 7^3\Big\}.
\end{equation}
\linebreak

En \textit{Python:}
\begin{lstlisting}[language = Python, caption = inicialización de la clase $\mathit{G7\_33}$]
G7_33 = GpnN(7,-3,3) #initialization
G7_33.generate_numbers()
\end{lstlisting}
La segunda línea genera todos los posibles números de la forma $ \sum_{k=-3}^{3} a_{k} 7^{k}$ y los guarda en una lista de números de tipo Número   (Number). En este caso, si quisiéramos visualizar los números asociados a $\mathit{G7\_{3}3}$, podemos verlos listados usando el método \textit{console\_printing} de la clase $\gpn$.


\begin{lstlisting}[language = Python, caption = Visualización de números en $\mathit{G7\_33}$]
G7_33.console_printing()

>>[0,0,0,0,0,0,0]
>>[0,0,0,0,0,0,1]
.
.
.      
>>[6,6,6,6,6,6,5]
>>[6,6,6,6,6,6,6]
\end{lstlisting}
\begin{remark}
	El orden de los números en $\gpn$ coincide con el orden inducido en $\R_+$ a través de la función \ref{Monna} definida como \textit{Monna map}.
\end{remark}
\subsubsection{Suma}
Para sumar dos números, usamos un algoritmo simple de complejidad lineal. Este algoritmo necesita de la inicialización del conjunto con el que se vaya a computar. Por ejemplo, si quisieramos sumar $ x =4324.321_5$ y $y=23.4123_5$ construimos el mínimo $\gpn$ que contenga a $x,y$. 

Dado que $x,y$ son de la  forma $x=\sum_{-3}^{3}a_k5^k$ y $y=\sum_{-4}^{1}b_k5^k$, el mínimo $\gpn\subset\Q_5$ que contiene a $x,y$ es $G5\_{3}4$. Así efectuamos $x+y$ en \textit{Python:}

\begin{lstlisting}[language = Python, caption = suma de números en $\mathit{G5\_34}$]
G5_34 = GpnN(5,-3,4) #initialization

x_digits = [4,3,2,4,3,2,1,0]
y_digits = [0,0,2,3,4,1,2,3]
x = Number   (5,-3,4,x_digits)
y = Number   (5,-3,4,y_digits)

x_plus_y = G5_34.p_sum   (x,y)
x_plus_y.show   ()
>>[4, 4, 0, 3, 2, 3, 3, 3]

\end{lstlisting}
Así $ 4324.321_5+23.4123_5=4403.2333_5$.
\begin{remark}
	En la línea 3 y 4 se agregaron ceros a los dígitos, ya que ${x=4324,321_5 = 4324,3210_5}$ y ${y=23,4123_5 =0023,4123_5}.$
\end{remark}

\subsubsection{Resta}
Análogo a la suma, usamos un algoritmo de complejidad lineal para efectuar la resta de dos números, como en el ejemplo anterior tomemos ${x =4324.321_5}$ y ${y=23.4123_5}$. De nuevo, construimos el mínimo $\gpn$ que contenga a $x,y$ y así  $\mathit{G5\_{3}4}$ será el subconjunto de $\Q_5$ que contiene $x,y$. Luego $x-y$ en \textit{Python:}


\begin{lstlisting}[language = Python, caption = resta de números en $\mathit{G5\_34}$]
G5_34 = GpnN(5,-3,4) #initialization

x_digits = [4,3,2,4,3,2,1,0]
y_digits = [0,0,2,3,4,1,2,3]

x = Number(5,-3,4,x_digits)
y = Number(5,-3,4,y_digits)

x_minus_y = G5_34.p_sub(x,y)
x_minus_y.show()
>>[4, 3, 0, 0, 4, 0, 3, 2]

\end{lstlisting}
Así $ 4324.321_5-23.4123_5=4300.4032_5$.

\subsubsection{Producto}
\begin{remark}
	\label{producto}    
	Supongamos que se multiplica un número de $m$  dígitos con otro que tiene $n$  dígitos. Luego, el producto tiene a lo máximo $m+n$ dígitos. Sin embargo, realizar estos cómputos, representan costos   (computacionalmente) a medida que $m,n$ se hacen grandes   (cosa que no pasa con la suma). Luego, nos restringimos a hacer multiplicaciones en el conjunto donde se esté trabajando.
	
	Para entender esta idea tomemos $x=3214.345356_7$ y $y = 53452.143_7$. Nótese que el mínimo $\gpn\subset\Q_7$ que contiene a $x,y$ es $\mathit{G7\_{4}6}$. Luego el producto será de la forma $\sum_{-6}^{4}a_k5^k$. Es decir, el producto tendrá \linebreak$-n+N+1=-   (-4)+6+1=11$ dígitos a lo más, y no $10+8=18$ que es la suma de la cantidad de dígitos de $x,y$ respectivamente.
\end{remark}

Siendo así, procedemos a calcular $x\cdot y$ en \textit{Python}:
\begin{lstlisting}[language = Python, caption = producto de números en $\mathit{G7\_46}$]
G7_46 = GpnN(7,-4,6) #initialization

x_digits = [0,3,2,1,4,3,4,5,3,5,6]
y_digits = [5,3,4,5,2,1,4,3,0,0,0]

x = Number(7,-4,6,x_digits)
y = Number(7,-4,6,y_digits)

x_by_y = G7_46.p_mul(x,y)
x_by_y.show()

>>[4, 0, 0, 6, 2, 0, 0, 5, 0, 0, 0]
\end{lstlisting}
Luego $3214.345356_7\cdot 53452.143_7=\cdots40062.005_7$

\subsubsection{División}
La observación \ref{producto}, hecha para el producto aplica para la división. Luego tomando $x,y$ como en el producto:

\begin{lstlisting}[language = Python, caption = división de números en $\mathit{G7\_46}$]
G7_46 = GpnN(7,-4,6) #initialization

x_digits = [0,3,2,1,4,3,4,5,3,5,6]
y_digits = [5,3,4,5,2,1,4,3,0,0,0]

x = Number(7,-4,6,x_digits)
y = Number(7,-4,6,y_digits)

x_div_y = G7_46.p_div(x,y)
x_div_y.show()
>>[5, 1, 3, 3, 0, 3, 3, 2, 3, 6, 2]
\end{lstlisting}
Así $\frac{3214.345356_7}{53452.143_7}=\cdots51330.332362_7$
\subsubsection{Aproximando Inversos}
Dado que la división puede llegar a ser un proceso infinito, se hace necesario restringir algunos coeficientes de su resultado. Entonces, hallar el inverso de un número $x$, básicamente corresponde a computar $\frac{1_p}{x}$. Luego, la aproximación del inverso de $x=49873210.54896_{11}$ está dada por:
\begin{lstlisting}[language = Python, caption = inversión de números en $\mathit{G11\_75}$]
G11_75 = GpnN(11,-7,5) #initialization

x_digits = [4,9,8,7,3,2,1,0,5,4,8,9,6]
x = Number(11,-7,5,x_digits)
x_inverse = G11_75.p_inverse(x)
x_inverse.show()
>>[7, 7, 7, 2, 7, 4, 6, 2, 0, 0, 0, 0, 0]
\end{lstlisting}
Luego $x^{-1}=\frac{1_{11}}{x}=\frac{1_{11}}{49873210.54896_{11}} = \cdots77727462_{11}$. 

\subsection{Árboles}
Denotamos por $M_n^m=\{x_0, x_1, \dots, x_{m^n}\}$ al árbol de $n$ niveles y $m$ ramas. Si fijamos $0<C<1$, podemos dotar a $M_n^m$ de una estructura ultramétrica y así, definimos la distancia en $M_n^m$ como $d   (x_i,x_j)=C^{N*}$, donde  $N*$ es el nivel del ancestro común más cercano a $x_i,x_j$. Así, $   (M_m^m,d)$ es un espacio ultramétrico.

Si tomamos $M_3^2=\{x_0, x_1,\dots, x_7\}$, tenemos que 
\begin{center}
	\begin{tabular}{||c c c c||} 
		\hline
		$x_i$ & $x_j$ & $N*$ & distancia \\ [0.5ex] 
		\hline\hline
		$x_0$ & $x_1$ & 2 & $C^2$ \\ 
		\hline
		$x_0$ & $x_2$ &1 & $C$ \\
		\hline
		$x_0$ & $x_3$ & 0 & 1 \\ 
		\hline
	\end{tabular}
\end{center}

\begin{figure}
	\caption{$M_n^2$}
	\begin{tikzpicture}[level/.style={sibling distance=40mm/#1}]
	\node [circle,draw]    (z){}
	child {node [circle,draw]    (a) {}
		child {node [circle,draw]    (b) {}
			child {node {$\vdots$}
				child {node     (d) {$x_1$}}
				child {node     (e) {$x_2$}}
			} 
			child {node {$\vdots$}}
		}
		child {node [circle,draw]    (g) {}
			child {node {$\vdots$}}
			child {node {$\vdots$}}
		}
	}
	child {node [circle,draw]    (j) {}
		child {node [circle,draw]    (k) {}
			child {node {$\vdots$}}
			child {node {$\vdots$}}
		}
		child {node [circle,draw]    (l) {}
			child {node {$\vdots$}}
			child {node    (c){$\vdots$}
				child {node   (o) {$x_{2^n-1}$}}
				child {node   (p) {$x_{2^n}$}
					child [grow=right] {node    (q) {$ $} edge from parent[draw=none]
						child [grow=right] {node    (q) {Nivel $n$} edge from parent[draw=none]
							child [grow=up] {node    (r) {$\vdots$} edge from parent[draw=none]
								child [grow=up] {node    (s) {Nivel 2} edge from parent[draw=none]
									child [grow=up] {node    (t) {Nivel 1} edge from parent[draw=none]
										child [grow=up] {node    (u) {Nivel 0} edge from parent[draw=none]}
									}
								}
							}
							%child [grow=down] {node    (v) {$O   (n \cdot \lg n)$}edge from parent[draw=none]}
						}
					}
				}
			}
		}
	};
	
	\end{tikzpicture}%}
\end{figure}

\begin{figure}[t]
	\caption{$M_3^2$}
	\centering
	\begin{tikzpicture}[level/.style={sibling distance=40mm/#1}]
	\node [circle,draw]    (z){}
	
	child {node [circle,draw]    (a) {}
		child {node [circle,draw]    (b) {}
			child {node {$x_1$}}
			child {node {$x_2$}}
		}
		child {node [circle,draw]    (g) {}
			child {node {$x_3$}}
			child {node {$x_4$}}
		}
	}
	child {node [circle,draw]    (j) {}
		child {node [circle,draw]    (k) {}
			child {node {$x_5$}}
			child {node {$x_6$}}
		}
		child {node [circle,draw]    (l) {}
			child {node {$x_7$}}
			child {node {$x_8$}}
		}
	};
	\end{tikzpicture}%}
\end{figure}


Luego, la estructura de árbol resulta natural al momento de representar los números $p$-ádicos tomando $C=\frac{1}{p}$ para $\gpn$. Luego, la implementación de un modelo para el conjunto $\mathit{G2\_22}\subset \Q_2$ (con $p=2$, $n=-2$ y $N=1$), está dado por:
\label{list_repreTree}
\begin{lstlisting}[language = Python, caption = Representación en árbol de $\mathit{G2\_21}$]
G2_21 = GpnN   (2,-2,1) #initialization
G2_21.generate_numbers()
G2_21.representation_tree()
\end{lstlisting}

\newpage
\newpage
Así, la función \texttt{representation\_tree()} mostrada en el anterior código, exportará una imagen en formato PNG de la siguiente forma:
\begin{center}	
	\includegraphics[scale=0.8]{img/G2_21}
\end{center}
Donde los nodos hoja representan números de la forma: $$x=\sum_{-1}^{2}a_k2^k = a_{-1}\cdot2^{-1}+a_0\cdot 2^0 + a_1\cdot 2 + a_2\cdot 2^2,$$

y que además satisfacen $2^{-2}\leq\norm{x}_2\leq 2$. Dado que los $a_k\in\{0,1\}$, con $k\in\{-1,0,1,2\}$, denotamos la elección de $a_k$ por las aristas del árbol, así, decir que $a_k\in\{\textbf{0},\textbf{\textcolor{red}{1}}\}$ será equivalente a decir que $a_k\in\{\textbf{Negro},\textbf{\textcolor{red}{Rojo}}\}$
en este ejemplo. 

Luego, si quisiéramos saber qué número es el que está ubicado en la parte derecha del árbol, de color \textbf{\textcolor{blue}{azul}}, sólo seguimos el camino propuesto por la imagen. Así, partiendo desde la raíz, tenemos que el camino hasta llegar a este nodo será $\textbf{Negro}\to\textbf{Negro}\to\textbf{Negro}\to\textbf{\textcolor{red}{Rojo}},$ es decir que el número es \textbf{000,}\textbf{\textcolor{red}{1}}$_2$ y tiene norma \textbf{\textcolor{blue}{$2^{-1}$}}; tal como lo indica la barra ubicada en la parte derecha de la imagen. Nótese que el camino desde la raíz denota $a_2\to a_1\to a_0\to a_{-1}$.
\begin{remark}
	En casos más generales, el entendimiento de la imagen será parecido. Luego, entre más grande sea el primo, más colores en las aristas habrá, así como habrá más normas y el recorrido de los elementos desde la raíz denotará $a_{-n}\to a_{n-1}\to\cdots\to a_0 \to \cdots \to a_{-N+1} \to a_{-N}.$
\end{remark}
\begin{figure}
	\begin{subfigure}{.5\textwidth}
		\centering
		\includegraphics[width=.8\linewidth]{img/G2_22}
		\caption{$\mathit{G2\_22}$}
		\label{fig:sfig1}
	\end{subfigure}%
	\begin{subfigure}{.5\textwidth}
		\centering
		\includegraphics[width=.8\linewidth]{img/G2_32}
		\caption{$\mathit{G2\_32}$}
		\label{fig:sfig2}
	\end{subfigure}
	\begin{subfigure}{.5\textwidth}
		\centering
		\includegraphics[width=.8\linewidth]{img/G2_33}
		\caption{$\mathit{G2\_33}$}
		\label{fig:sfig1}
	\end{subfigure}%
	\begin{subfigure}{.5\textwidth}
		\centering
		\includegraphics[width=.8\linewidth]{img/G2_43}
		\caption{$\mathit{G2\_43}$}
		\label{fig:sfig1}
	\end{subfigure}%
	\caption{Representación en árbol de los números $2$-ádicos.}
	\label{2adics_trees}
\end{figure}
  	%\chapter{Laplaciano sobre árboles}
\label{chapter_3}

\section{Matriz Laplaciana}
Dado un grafo simple G con $n$ vértices, definimos la matriz Laplaciana $L\in \bb{M}_n(\R)$ como :
$$ L = D-A,$$
donde $D,A$ son las matrices de grados e incidencia del grafo, respectivamente. Así, se tiene que las entradas de esta matriz estarán dadas por:
$$
L_{i, j}\coloneqq \left\{\begin{array}{ll}
\operatorname{grado}\left(v_{i}\right) & \text { si } i=j, \\
-1 & \text { si } i \neq j \text { y } v_{i} \text { es adyacente a } v_{j}, \\
0 & \text { e.o.c. }
\end{array}\right.$$
Esta matriz tiene propiedades, de las cuales nombraremos  las que nos interesan:
\begin{itemize}
	\item[$\diamond$] $L$ es simétrica;
	\item[$\diamond$] $L$ es semidefinida-positiva (sus autovalores son positivos);
	\item[$\diamond$] La suma por filas o columnas es $0$.
\end{itemize}
Esta matriz es llamada así, puesto que coincide con el laplaciano discreto. Ver por ejemplo \cite{networksAI}. Razón por la cual podemos hacer una interpretación del mismo.

\section{Operador Laplaciano discreto}
Supongamos una función $\vec{\phi}(t)$ que describe la distribución de calor dentro de un grafo en un tiempo dado. Así, $\phi_i(t)$ es el calor en el nodo $i$ en el tiempo $t$. Luego, el calor transferido entre los nodos $i$ y $j$ es directamente proporcional a la difencia de calor entre los mismos \footnote{Ley de enfriamiento de Newton}, es decir, proporcional a $\phi_i(t)-\phi_j(t)$. Así, para una constante $k$ (conocida como constante de capacidad de calor) esto se representa de la siguiente forma:

\[
\begin{aligned}
\frac{d \phi_{i}(t)}{d t} &=-k \sum_{j} A_{i j}\left(\phi_{i}(t)-\phi_{j}(t)\right) \\
&=-k\left(\phi_{i}(t) \sum_{j} A_{i j}-\sum_{j} A_{i j} \phi_{j}(t)\right) \\
&=-k\left(\phi_{i}(t) \operatorname{grado}\left(v_{i}\right)-\sum_{j} A_{i j} \phi_{j}(t)\right) \\
&=-k \sum_{j}\left(\delta_{i j} \operatorname{grado}\left(v_{i}\right)-A_{i j}\right) \phi_{j}(t) \\
&=-k \sum_{j}\left(\ell_{i j}\right) \phi_{j}(t),
\end{aligned}
\]
que se reescribe como:
\[
\begin{aligned}
\frac{d \vec{\phi}(t)}{d t} &=-k(D-A) \vec{\phi}(t) \\
&=-k L \vec{\phi}(t),
\end{aligned}
\]
es decir:
\begin{equation}\label{heateq}
\frac{d \vec{\phi}(t)}{d t}+k L \vec{\phi}(t)=0.
\end{equation}
Nótese que la ecuación \ref{heateq} tiene la forma de la ecuación de calor, donde la matriz $L$ toma el valor de $\nabla^2$. Por esta razón, hablamos de $L$ como el Laplaciano de un grafo.

Además, la ecuación \ref{heateq} es un sistema de ecuaciones de primer orden cuya solución para $\vec{\phi}(t)$ es una combinación lineal de autovectores de $L$(con autovalores $(\lambda_i)$):

$$\sum_{i=1}^{\abs{V}}c_i(t)\vec{v_i},\text{ con $V$ es el conjunto de vértices.}$$
Reemplazando en la ecuación \ref{heateq}, tenemos:

\begin{align*}
\frac{d\left(\sum_{i} c_{i}(t) \vec{v}_{i}\right)}{d t}+k L\left(\sum_{i} c_{i}(t) \vec{v}_{i}\right) &=0 \\
\sum_{i}\left[\frac{d c_{i}(t)}{d t} \vec{v}_{i}+k c_{i}(t) L \vec{v}_{i}\right] &=\\
\sum_{i}\left[\frac{d c_{i}(t)}{d t} \vec{v}_{i}+k c_{i}(t) \lambda_{i} \vec{v}_{i}\right] &;\\
\Rightarrow \frac{d c_{i}(t)}{d t}+k \lambda_{i} c_{i}(t) &=0,
\end{align*}
que es una ecuación homogénea, cuya solución está dada por:
$$c_i(t)=c_i(0)e^{-k\lambda_it}.$$
\begin{definition}[Árbol de Cayley]
	Un árbol en el cual todos sus nodos que no son hojas tiene un número constante de ramas $n$, es llamado \textit{$n$-árbol de Cayley.}
\end{definition}

\begin{center}
	\begin{figure}[h]
		\begin{tikzpicture}[x=0.75pt,y=0.75pt,yscale=-1,xscale=1]
		%uncomment if require: \path (0,300); %set diagram left start at 0, and has height of 300
		
		%Straight Lines [id:da5842217325180374] 
		\draw    (198.71,51.5) -- (199,102) ;
		%Straight Lines [id:da7276215169992284] 
		\draw    (160,140) -- (199,102) ;
		%Straight Lines [id:da28480116898973207] 
		\draw    (240,140) -- (199,102) ;
		%Straight Lines [id:da9834829913476775] 
		\draw    (312.71,100) -- (371.71,100) ;
		%Straight Lines [id:da9583545055171534] 
		\draw    (312.71,100.5) -- (290,140) ;
		%Straight Lines [id:da9799941223701523] 
		\draw    (290,59.5) -- (312.71,100.5) ;
		%Straight Lines [id:da38279622647558265] 
		\draw    (394,61.17) -- (371.71,100) ;
		%Straight Lines [id:da07459517917616099] 
		\draw    (371.71,100) -- (394,140.17) ;
		%Straight Lines [id:da3965956580664982] 
		\draw    (489,51.5) -- (489,79.5) ;
		%Straight Lines [id:da016839034481049175] 
		\draw    (472,99.17) -- (489,79.5) ;
		%Straight Lines [id:da7562850854840744] 
		\draw    (509,99.17) -- (489,79.5) ;
		%Straight Lines [id:da8034784066626488] 
		\draw    (452.67,90.17) -- (472,99.17) ;
		%Straight Lines [id:da8680080185249526] 
		\draw    (461.67,121.17) -- (472,99.17) ;
		%Straight Lines [id:da4124727277691931] 
		\draw    (509,99.17) -- (532.67,90.17) ;
		%Straight Lines [id:da2855392219619475] 
		\draw    (509,99.17) -- (519.67,123.17) ;
		
		
		
		
		\end{tikzpicture}
		\caption{3-árboles de Cayley}
	\end{figure}
\end{center}

Nótese que la definición anterior viene dada por la definición de grafo de Cayley. Además, dado que los números $p$-ádicos se pueden representar en $p$-árboles de Cayley, presentamos algunas ecuaciones.


%Nótese que $0<\omega(x,y)\leq 1$ para todo $x,y \in \Qp$.
\newpage
\subsection{Matriz de transición}
En el anterior capítulo definimos el grupo $GpnN$ (\ref{GpnN}):
\begin{equation*}
GpnN := \Big\{x\in \Qp : x = \sum_{k=-N}^{-n} a_{k} p^{k} \text{, $a_k\in \{0,\cdots,p-1\}$ y } p^{n}\leq \pnorm{x}\leq p^N\Big \}.
\end{equation*}
Nótese que $GpnN$ tiene cardinal $p^{N-n+1}$(recordemos que $n\leq0$), luego $GpnN$ es finito y así, podemos indexar sus elementos, sean $\{x_1,\dots,x_{p^{N-n}},x_{p^{N-n+1}}\}$ y corresponderlos inyectivamente con el conjunto $\{1,\dots,p^{N-n},p^{N-n+1}\}$ a través de la siguiente función:
\begin{align*}
&l\colon\{1,\dots,p^{N-n+1}\}\to GpnN \Rightarrow l(i)=x_i.
%	&l^{-1}\colon GpnN\to \{1,\dots,p^{N-n+1}\} \Rightarrow l^{-1}(x)= l^{-1}\big(\sum_{k=N}^{n}a_kp^{-k}\big)=1+p^{-1}\sum_{k=N}^{n}a_kp^k,
\end{align*}

Definimos una \textit{matriz de réplica} a la matriz $\textbf{Q   }=(Q_{ab}) $ de tamaño $p^{N-n+1}\times p^{N-n+1}$, como un operador en el espacio de funciones del conjunto $\{1,\dots,p^{N-n+1}\}$ tal que 
\begin{equation}\label{Q}
Q_{ij}=\rho(\pnorm{l(i)-l(j)}),
\end{equation}
donde $\rho$ es una función que depende de la distancia $p$-ádica entre $l(i)$ y $l(j)$. 
Luego, la estructura de la matriz $\textbf{Q}$ con $p=2$ es de tipo Parisi (\cite{Parisi-2000}):
$$
\boldsymbol{Q}=\left(\begin{array}{lllllllll}
0 & q_{1} & q_{2} & q_{2} & q_{3} & q_{3} & q_{3} & q_{3} & \dots \\
q_{1} & 0 & q_{2} & q_{2} & q_{3} & q_{3} & q_{3} & q_{3} & \dots \\
q_{2} & q_{2} & 0 & q_{1} & q_{3} & q_{3} & q_{3} & q_{3} & \dots \\
q_{2} & q_{2} & q_{1} & 0 & q_{3} & q_{3} & q_{3} & q_{3} & \dots \\
q_{3} & q_{3} & q_{3} & q_{3} & 0 & q_{1} & q_{2} & q_{2} & \dots \\
q_{3} & q_{3} & q_{3} & q_{3} & q_{1} & 0 & q_{2} & q_{2} & \dots \\
q_{3} & q_{3} & q_{3} & q_{3} & q_{2} & q_{2} & 0 & q_{1} & \dots \\
q_{3} & q_{3} & q_{3} & q_{3} & q_{2} & q_{2} & q_{1} & 0 & \dots\\
&  &  &  & \vdots & & &  & \ddots
\end{array}\right)
$$
Luego, definimos la \textit{matriz de transición } $W$, a partir de la matriz de réplica como:
\[   
W_{ij} =
\begin{cases}
Q_{ij} & i\neq j,\\
-\sum_{\gamma \neq i}^{N-n+1}Q_{i\gamma} &i=j.\\

\end{cases}
\]

Dado $GpnN\subset \Qp$, sabemos que tiene una representación natural de $(N-n+1)$-árbol de Cayley. Para entender mejor la representación, tomemos como ejemplo a $G2\_11$; nótese que $\abs{G2\_11}=8$, por lo tanto, podemos indexar sus elementos en $\{1,\dots,8\}$, ver por ejemplo \ref{G2_11}. Luego $G2\_11$ tendrá una matriz de transición $W$ con la siguiente estructura (matriz de Parisi):

\[
W=
\left[
\begin{array}{c|c}
\begin{array}{c|c}
\begin{array}{cc}
w_0 & w_1 \\

w_1 & w_0
\end{array} & w_2 \\
\hline
w_2 & \begin{array}{cc}
w_0 & w_1 \\

w_1 & w_0
\end{array}

\end{array} & w_3 \\
\hline
w_3 & \begin{array}{c|c}
\begin{array}{cc}
w_0 & w_1 \\

w_1 & w_0
\end{array}& w_2 \\
\hline
w_2 &\begin{array}{cc}
w_0 & w_1 \\

w_1 & w_0
\end{array}
\end{array}
\end{array}
\right]
\]

\begin{figure}
	\caption{$G2\_11$}
	\label{G2_11}
	\centering
	\begin{tikzpicture}[level/.style={sibling distance=40mm/#1}]
	\node [circle,draw] (z){$w_3$}
	
	child {node [circle,draw] (a) {$w_2$}
		child {node [circle,draw] (b) {$w_1$}
			child {node {$1$}}
			child {node {$2$}}
		}
		child {node [circle,draw] (g) {$w_1$}
			child {node {$3$}}
			child {node {$4$}}
		}
	}
	child {node [circle,draw] (j) {$w_2$}
		child {node [circle,draw] (k) {$w_1$}
			child {node {$5$}}
			child {node {$6$}}
		}
		child {node [circle,draw] (l) {$w_1$}
			child {node {$7$}}
			child {node {$8$}}
		}
	};
	\end{tikzpicture}%}
\end{figure}
En general, esta matriz muestra las interacciones entre los nodos hojas (que es donde inicialmente se encuentran los números $p$-ádicos en la representación dada en el capítulo anterior). Además, dado que la ecuación \ref{Q} establece que la matriz depende de la distancia asociada a dos estados $i,j\in\{1, \dots,p^{N-n+1}\}$, damos un ejemplo de tales matrices $Q_{ij}$, tomando $\rho$ como sigue:
\begin{equation}
\label{f_prob}
Q_{ij}=\rho(\pnorm{l(i)-l(j)})= \frac{C}{\pnorm{x-y}^\alpha + 1},\text{ $x,y\in GpnN$}.
\end{equation}

%y representa la tasa de transición de $x_i$ a $x_j$. La construcción de esta matriz tiene motivación en Biología, y más específicamente en el estudio de proteinas (ver \cite{Avetisov-1999,Avetisov-2002,Parisi-2000}). Siendo así, podemos definir la ecuación maestra.

\begin{figure}
	\begin{subfigure}{.6\textwidth}
		\centering
		\includegraphics[width=.8\linewidth]{img/matrix/matrixG2_22}
		\caption{Matriz de $\mathit{G2\_22}$}
		\label{fig:sfig1}
	\end{subfigure}%
	\begin{subfigure}{.6\textwidth}
		\centering
		\includegraphics[width=.8\linewidth]{img/matrix/matrixG2_32}
		\caption{Matriz de $\mathit{G2\_32}$}
		\label{fig:sfig2}
	\end{subfigure}
	\begin{subfigure}{.6\textwidth}
		\centering
		\includegraphics[width=.8\linewidth]{img/matrix/matrixG2_33}
		\caption{Matriz de $\mathit{G2\_33}$}
		\label{fig:sfig1}
	\end{subfigure}%
	\begin{subfigure}{.6\textwidth}
		\centering
		\includegraphics[width=.8\linewidth]{img/matrix/matrixG2_43}
		\caption{Matriz de $\mathit{G2\_43}$}
		\label{fig:sfig1}
	\end{subfigure}%
	
	\caption{Matrices de Parisi asociadas a los estados de las figuras \ref{2adics_trees} con $\alpha=2$ y $C=3$}
	\label{2adic_matrices}
\end{figure}
\pagebreak

\subsection{Ecuación de ultradifusión}
Análogamente a como se estableció la ecuación de calor sobre grafos(\ref{heateq}), definimos la ecuación maestra como sigue:
\begin{equation}
\label{ME}
\frac{du_i(t)}{dt} = \sum_{j\neq i}W_{ji}u_j(t) - \sum_{j\neq i}W_{ij}u_i(t), 
\end{equation}
donde $u_i$ es la probabilidad de transición, es decir, la probabilidad de pasar del estado $i$ al estado $j$ en el tiempo $t$.

Además, como hicimos con la ecuación \ref{heateq}, podemos llevar la ecuación \ref{ME} a la forma:
$$\frac{du(t)}{dt} = Wu(t).$$
Cuya solución, estará dada de manera estándar como una combinación lineal de autovectores de $W$:
$$u(t)=\sum_{i=1}^{p^{N-n+1}}c_i(t)v_i.$$
Con:
$$c_i(t)=c_i(0)e^{\lambda_it}.$$

Así, la solución numérica del problema anterior con distintas condiciones iniciales está dada por las figuras \ref{IC-const}, \ref{IC-random} y \ref{IC-bell}.               

\begin{figure}
	\begin{subfigure}{.5\textwidth}
		\centering
		\includegraphics[width=.8\linewidth]{img/solutions/ones_0}
		\caption{$t=0$}
		\label{fig:sfig1}
	\end{subfigure}%
	\begin{subfigure}{.5\textwidth}
		\centering
		\includegraphics[width=.8\linewidth]{img/solutions/ones_1}
		\caption{$t=3.33$}
		\label{fig:sfig1}
	\end{subfigure}
	\begin{subfigure}{.5\textwidth}
		\centering
		\includegraphics[width=.8\linewidth]{img/solutions/ones_2}
		\caption{$t=6.67$}
		\label{fig:sfig1}
	\end{subfigure}%
	\begin{subfigure}{.5\textwidth}
		\centering
		\includegraphics[width=.8\linewidth]{img/solutions/ones_3}
		\caption{$t=10$}
		\label{fig:sfig1}
	\end{subfigure}
	\begin{subfigure}{1.05\textwidth}
		\centering
		\includegraphics[width=.8\linewidth]{img/solutions/ones_bar}
		%\caption{$t=10$}
		\label{fig:sfig1}
	\end{subfigure}%
	\caption{Comportamiento de la solución en distintos $t\in[0,10]$ para la condición inicial constante $u(0)=\vec{1}$ en $\mathit{G2\_05}.$}
	\label{IC-const}
\end{figure}



\begin{figure}
	\begin{subfigure}{.5\textwidth}
		\centering
		\includegraphics[width=.8\linewidth]{img/solutions/random_0}
		\caption{$t=0$}
		\label{fig:sfig1}
	\end{subfigure}%
	\begin{subfigure}{.5\textwidth}
		\centering
		\includegraphics[width=.8\linewidth]{img/solutions/random_1}
		\caption{$t=3.33$}
		\label{fig:sfig1}
	\end{subfigure}
	\begin{subfigure}{.5\textwidth}
		\centering
		\includegraphics[width=.8\linewidth]{img/solutions/random_2}
		\caption{$t=6.67$}
		\label{fig:sfig1}
	\end{subfigure}%
	\begin{subfigure}{.5\textwidth}
		\centering
		\includegraphics[width=.8\linewidth]{img/solutions/random_3}
		\caption{$t=10$}
		\label{fig:sfig1}
	\end{subfigure}
	\begin{subfigure}{1.05\textwidth}
		\centering
		\includegraphics[width=.8\linewidth]{img/solutions/random_bar}
		%\caption{$t=10$}
		\label{fig:sfig1}
	\end{subfigure}%
	\caption{Comportamiento de la solución en distintos $t\in[0,10]$ para la condición inicial $u(0)$ aleatoriamente distribuida, con cada entrada en $[0,1]$, en $\mathit{G2\_05}$.}
	\label{IC-random}
\end{figure}

\begin{figure}
	\begin{subfigure}{.5\textwidth}
		\centering
		\includegraphics[width=.8\linewidth]{img/solutions/normal_0}
		\caption{$t=0$}
		\label{fig:sfig1}
	\end{subfigure}%
	\begin{subfigure}{.5\textwidth}
		\centering
		\includegraphics[width=.8\linewidth]{img/solutions/normal_1}
		\caption{$t=3.33$}
		\label{fig:sfig1}
	\end{subfigure}
	\begin{subfigure}{.5\textwidth}
		\centering
		\includegraphics[width=.8\linewidth]{img/solutions/normal_2}
		\caption{$t=6.67$}
		\label{fig:sfig1}
	\end{subfigure}%
	\begin{subfigure}{.5\textwidth}
		\centering
		\includegraphics[width=.8\linewidth]{img/solutions/normal_3}
		\caption{$t=10$}
		\label{fig:sfig1}
	\end{subfigure}
	\begin{subfigure}{1.05\textwidth}
		\centering
		\includegraphics[width=.8\linewidth]{img/solutions/normal_bar}
		%\caption{$t=10$}
		\label{fig:sfig1}
	\end{subfigure}%
	\caption{Comportamiento de la solución en distintos $t\in[0,10]$ para la condición inicial $u(0)$ en forma de campana de Gauss, en $\mathit{G2\_33}$.}
	\label{IC-bell}
\end{figure}
\end{document}